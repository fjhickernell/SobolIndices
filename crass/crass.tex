% Template article for preprint document class `elsart'
% SP 2001/01/05
% Modified CG (ESME) for Model 3, single column, 2 titles, abstract/r�sum�,
%  and 2 sets of keywords - 07.01.03 - file called Maths-English.tex
% English Version for Mathematics (CRAS series 1)
% Revamped, CG, 17.08.04, adding header, dates, and presenter

\documentclass{elsart3-1}

% Use the option doublespacing or reviewcopy to obtain double line spacing
% \documentclass[doublespacing]{elsart}

% if you use PostScript figures in your article
% use the graphics package for simple commands
% \usepackage{graphics}
% or use the graphicx package for more complicated commands
% \usepackage{graphicx}
% or use the epsfig package if you prefer to use the old commands
% \usepackage{epsfig}

% The amssymb package provides various useful mathematical symbols
\usepackage{amsmath,amssymb}

\usepackage[english,francais]{babel}

%ENVIRONMENTS THEOREMS...
% These are predefined, and follow the numbering system used in the journal!
%English
\newtheorem{theorem}{Theorem}[section]
\newtheorem{lemma}[theorem]{Lemma}
\newtheorem{e-proposition}[theorem]{Proposition}
\newtheorem{corollary}[theorem]{Corollary}
\newtheorem{e-definition}[theorem]{Definition\rm}
\newtheorem{remark}{\it Remark\/}
\newtheorem{example}{\it Example\/}
%French
\newtheorem{theoreme}{Th\'eor\`eme}[section]
\newtheorem{lemme}[theoreme]{Lemme}
\newtheorem{proposition}[theoreme]{Proposition}
\newtheorem{corollaire}[theoreme]{Corollaire}
\newtheorem{definition}[theoreme]{D\'efinition\rm}
\newtheorem{remarque}{\it Remarque}
\newtheorem{exemple}{\it Exemple\/}
\renewcommand{\theequation}{\arabic{equation}}
\setcounter{equation}{0}

\newcommand{\fudge}{\fC}
\newcommand{\dtf}{\textit{\doubletilde{f}}}
\newcommand{\cube}{[0,1)^s}
\newcommand{\rf}{\mathring{f}}
\newcommand{\rnu}{\mathring{\nu}}
\newcommand{\natm}{\naturals_{0,m}}
\newcommand{\wcS}{\widecheck{S}}
\newcommand{\tol}{\text{tol}}
\newcommand{\bvec}[1]{\boldsymbol{#1}}
\newcommand{\vx}{\bvec{x}}
\newcommand{\vi}{\bvec{i}}
\newcommand{\vj}{\bvec{j}}
\newcommand{\ve}{\bvec{e}}
\newcommand{\vk}{\bvec{k}}
\newcommand{\vz}{\bvec{z}}
\newcommand{\dif}{\mathsf{d}}
\newcommand{\hf}{\hat{f}}
\newcommand{\hS}{\widehat{S}}
\newcommand{\tS}{\widetilde{S}}
\newcommand{\tf}{\tilde{f}}
\newcommand{\fC}{\mathfrak{C}}
\newcommand{\homega}{\widehat{\omega}}
\newcommand{\wcomega}{\mathring{\omega}}
\newcommand{\vzero}{\bvec{0}}
\newcommand{\integers}{\mathbb{Z}}
\newcommand{\naturals}{\mathbb{N}}
\newcommand{\ip}[3][{}]{\ensuremath{\left \langle #2, #3 \right \rangle_{#1}}}

\def\abs#1{\ensuremath{\left \lvert #1 \right \rvert}}

%%%%%%%%%%%%%%%%%%%%%%%%%%%%%%%%
%% GUILLEMETS (FRENCH QUOTES) %%
%%%%%%%%%%%%%%%%%%%%%%%%%%%%%%%%
\def\og{\leavevmode\raise.3ex\hbox{$\scriptscriptstyle\langle\!\langle$~}}
\def\fg{\leavevmode\raise.3ex\hbox{~$\!\scriptscriptstyle\,\rangle\!\rangle$}}


\journal{the Acad\'emie des sciences}
\begin{document}
% place in the next line the header (rubrique) chosen for your article,
% if you know it (you can also have 2, format : Header1/Header2
\centerline{}
\begin{frontmatter}

% Title, authors and addresses

% use the thanksref command within \title, \author or \address for footnotes;
% use the ead command for the email address,
% and the form \ead[url] for the home page:
% \title{Title\thanksref{label1}}
% \thanks[label1]{}
% \author{Name\thanksref{label2}}
% \ead{email address}
% \ead[url]{home page}
% \thanks[label2]{}
% \address{Address\thanksref{label3}}
% \thanks[label3]{}
\selectlanguage{english}
\title{Replicated designs based on Sobol' sequences to estimate main effects of model inputs}
%\title{On the replication of Sobol' sequences to estimate main effects of model inputs}


% use optional labels to link authors explicitly to addresses:
% \author[label1,label2]{}
% \address[label1]{}
% \address[label2]{}
% The [label1] can be suppressed if there is only one address for all authors

\selectlanguage{english}
\author[authorlabel1]{Laurent Gilquin},
\ead{laurent.gilquin@inria.fr}
\author[authorlabel2]{Llu\'{i}s Antoni Jim\'{e}nez Rugama},
%\ead{ljimene1@hawk.iit.edu}
\author[authorlabel3]{Elise Arnaud},
\author[authorlabel2]{Fred J. Hickernell},
\author[authorlabel4]{Herv\'{e} Monod},
\author[authorlabel3]{Cl\'{e}mentine Prieur}

\address[authorlabel1]{Inria Grenoble - Rh\^{o}ne-Alpes, Inovall\'{e}e, 655 avenue de l'Europe, 38330 Montbonnot}
\address[authorlabel2]{TO FILL}
\address[authorlabel3]{Univ. Grenoble Alpes, Jean Kunzmann Laboratory, F-38000 Grenoble, France \\
CNRS, LJK, F-38000 Grenoble, France, Inria}
\address[authorlabel4]{MaIAGE, INRA, Universit\'{e} Paris-Saclay, 78350 Jouy-En-Josas, France}

% If you know the dates of reception, and acceptation you can put them now;
%  idem the name of the person presenting the Note

\medskip
\begin{center}
{\small Received *****; accepted after revision +++++\\
Presented by �����}
\end{center}

\begin{abstract}
\selectlanguage{english}
% Text of abstract in English
Your abstract in English here.
{\it To cite this article: A.
Name1, A. Name2, C. R. Acad. Sci. Paris, Ser. I 340 (2005).}

\vskip 0.5\baselineskip

\selectlanguage{francais}
% Text of abstract in French
\noindent{\bf R\'esum\'e} \vskip 0.5\baselineskip \noindent
{\bf Here is the title in French. }
Your resume in French here.
{\it Pour citer cet article~: A. Name1, A. Name2, C. R. Acad. Sci.
Paris, Ser. I 340 (2005).}

\end{abstract}
\end{frontmatter}

\selectlanguage{english}

\section{Introduction}
Mathematical models used in various fields often involve a substantial number of poorly known parameters. The effect of these parameters on the output of the model can be assessed through sensitivity analysis. Global sensitivity analysis methods are useful tools to identify the parameters having the most influence on the output. A well known approach is the variance based method introduced by Sobol' \cite{sobol'}. This method estimates sensitivity indices called Sobol' indices that summarize the influence of each model's input. In particular, one can distinguish first-order indices that estimate the main effect of each input.

The procedure proposed by Sobol' to estimate first-order Sobol' indices and its improvements (see Saltelli \cite{Saltelli} for an exhaustive survey) all suffer from a prohibitive cost of model evaluations that grows with respect to the input space dimension. %An elegant solution to reduce this cost relies on the construction of replicated designs introduced by McKay \cite{mckay}.  
 An elegant solution to reduce this cost relies on the construction of particular designs of experiments called replicated designs. The notion of replicated designs was first introduced by McKay through its introduction of replicated Latin Hypercubes \cite{mckay}. 
Here, we define replicated designs in a wider framework:

\begin{definition}%[Replicated designs]
\label{rep.designs}
Denote by $\vx \in [0,1)^s$ a point with components $x_1,\dots,x_s$. Set $\mathcal{D}=\{1,\dots,s\}$. Let $\mathcal{P}=\{\vx_i\}_{i=0}^{n-1}$ and $\mathcal{P}'=\{{\vx'}_i\}_{i=0}^{n-1}$ be two point sets,
$\vx_i,\vx'_i\in[0,1)^{s}$. Let $\mathcal{P}^u=\{\vx_{i,u}\}_{i=0}^{n-1}$ (resp. $\mathcal{P}'^u$), $u \subsetneq \mathcal{D}$, denote the subset of dimensions of $\mathcal{P}$ (resp. $\mathcal{P}'$) indexed by $u$. We say that $\mathcal{P}$ and $\mathcal{P}'$ are two replicated designs of order $r$ if:\\
$\forall \ u \subsetneq \mathcal{D}$ such that $|u|=r$, $\mathcal{P}^u$ and $\mathcal{P}'^u$ are the same point set in $[0,1)^r$.
\end{definition}

With the construction of only two replicated designs of order $1$, the replication procedure introduced and studied in \cite{mara,tissot} allows to estimate all first-order Sobol' indices. 

This procedure has the major advantage of reducing drastically the estimation cost as the number of runs (one design $\mathcal{P}$ of size $n$  and its replication $\mathcal{P}'$) becomes independent of the input space dimension. However, Sobol' indices estimates may still not be accurate enough if the input space is not properly explored.
\bigskip

In this note, we propose to construct two replicated point sets based on Sobol' sequences that identify as two space-filling replicated designs of order $1$. These two designs insure that the input space is properly explored and can then be used within the replication procedure to estimate all main-effects of a numerical model. We first provide backgrounds on digital sequences, then we present two iterative approaches to construct the two replicated point sets. 

\section{Digital sequences background}

\subsection{Preliminaries}

The introduction of digital nets and sequences has its origin in numerical integration and uniform distribution. Explicit constructions of digital nets and sequences have been developed notably by Sobol' and Niederreiter \cite{niderreiter} into the overarching notion of $(t,m,s)$-nets and $(t,s)$-sequences. A comprehensive theory of $(t,m,s)$-net and $(t,s)$-sequences was first formulate by Niederreiter. 

$(t,m,s)$-nets are defined as point sets in $\cube$ whose quality is measured by the parameter $t$, called $t$-value. They are defined as follows:
\begin{definition}%[$(t,m,s)-net$]
Let $\mathcal{A}$ be the set of all elementary intervals $A\in\cube$ where $A=\prod_{j=1}^s [\alpha_jb^{-\gamma_j},(\alpha_j+1)b^{-\gamma_j})$, with integers $s\geq 1$, $b\geq 2$, $\gamma_j\geq 0$, and $b^{\gamma_j}>\alpha_j\geq 0$. For $m\geq t\geq 0$, the point set $\mathcal{P}\in\cube$ with $b^m$ points is a $(t,m,s)-net$ in base $b$ if every $A$ with volume $b^{t-m}$ contains $b^t$ points of $\mathcal{P}$.
\end{definition}

A $(t,m,s)$-net is defined such that all elementary intervals of volume at least $b^{t-m}$ will enclose a proportional number of points. The most evenly spread nets are $(0,m,s)$-nets, since each elementary interval of the smallest volume possible, $b^{-m}$, contains exactly one point. %Note that if $t'>t$, $(t,m,s)$-nets are always $(t',m,s)$-nets.

$(t,s)$-sequences are the infinite analogues of $(t,m,s)$-nets. They are defined as follows:
\begin{definition}%[$(t,s)$-sequence]
For integers $s\geq 1$, $b\geq 2$, and $t\geq 0$, the sequence $\{\vx_i\}_{i\in\mathbb{N}_0}$ is a $(t,s)$-sequence in base $b$, if for every set $\mathcal{P}_{\ell,m}=\{\vx_i\}_{i=\ell b^m}^{(\ell+1)b^m-1}$ with $\ell\geq 0$ and $m\geq t$, $\mathcal{P}_{\ell,m}$ is a $(t,m,s)$-net in base $b$.
\end{definition}
\bigskip

All $(t,m,s)$-nets and $(t,s)$-sequences relevant for applications are based on a digital construction scheme over a finite field $\mathbb{F}_b$ with $b$ a prime power number. Digital nets and sequences referred to these structures. 
\bigskip

<<<<<<< HEAD
The construction scheme of a $s$-dimensional digital sequence requires the knowledge of the base $b$ and a set of $s$ full rank upper triangular matrices of infinite size called generating matrices. These generating matrices are recursively constructed through primitive polynomials and initial directional numbers (see \cite{niederreiter} for further details). Denote by $C_1,\dots,C_s$ $s$ generating matrices over $\mathbb{F}_b:=\{0,\dots,b-1\}$. For $m>0$ and $j \in \{1,\dots,s\}$, we note $C_j^{m}$ the $m \times m$ upper left block of $C_j$. The first $b^m$ points $\{\vx_0,\dots,\vx_{b^m-1}\}$ of the digital sequence are constructed as follows:\\
=======
The construction scheme of a $s$-dimensional digital sequence requires the knowledge of the base $b$ and a set of $s$ full rank upper triangular matrices of infinite size called generating matrices. These generating matrices are recursively constructed through primitive polynomials and initial directional numbers (see \cite{niederreiter} for further details). We denote by $C_1,\dots,C_s$ the $s$ generating matrices over $\mathbb{F}_b$ of the $s$-dimensional digital sequence. For $m>0$ and $j \in \{1,\dots,s\}$, we note $C_j^{m}$ the $m \times m$ upper left block of $C_j$. The first $b^m$ points $\{\vx_0,\dots,\vx_{b^m-1}\}$ of the digital sequence are constructed as follows:\\

For each $i=0,\dots,b^m-1$, the point $\vx_i = (x_{i,1},\dots,x_{i,s})^\intercal$ of the sequence is obtained dimension-wise by:
\begin{equation}
\label{dig.net.eq.}
(x_{i,j,1},x_{i,j,2},\dots)^\intercal = C_j^{m} \vi,\qquad j= 1,\dots,s\, ,
\end{equation}
where $x_{i,j} = \sum_{k \geq 1}x_{i,j,k}b^{-k}$ and $\vi = (i_{0},\dots,i_{m-1})^\intercal$ is the b-ary decomposition of $i$.
\bigskip

Remark that for any $m' > m$, the matrix $C_j^m$ is embedded in the upper left of matrices $C_j^{m'}$. Resulting from this embedding, the construction of a digital sequence can be performed iteratively.

The concept of digital sequences can be linked to the definition of replicated designs. To do so, we introduce the following notion:
\begin{definition}
Two digital sequences $\{\vx_i\}_{i\in\mathbb{N}_0}$ and $\{{\vx'}_i\}_{i\in\mathbb{N}_0}$ are said \emph{digitally replicated} of order $r$ if for all $m\geq 0$, $\{{\vx}_i\}_{i=0}^{b^m-1}$ and $\{{\vx'}_i\}_{i=0}^{b^m-1}$ are two replicated designs of order $r$.
\end{definition}
 
\subsection{Sobol' sequences}

Sobol' sequences are digital sequences constructed in base $2$. These sequences are attractive for their easy implementation and optimized properties \cite{kuo2}. Sobol' sequences are particularly interesting for having full rank upper triangular generating matrices over $\mathbb{F}_2$. This property leads to the following lemma:
\begin{lemma}\label{Sobol_replicated}
All Sobol' sequences are digitally replicated of order 1.
\end{lemma}
\begin{pf*}{\textit{Proof}.}
Consider any two s-dimensional Sobol' sequences generated respectively by the matrices $C_1,\dots,C_s$ and $C'_1,\dots,C'_s$. Since these matrices are upper triangular, to compute the first $2^m$ points one only needs the knowledge of the $m\times m$ upper left blocks $C_j^m$ and ${C'}_j^m$, $j \in \{1,\dots,s\}$. Because any two matrices $C^{m}_j$ and ${C'}^{m}_j$ are square and full rank, the products $C^{m}_j\vi$ and ${C'}^{m}_j\vi$ are one-to one and onto, for all $i=0,\dots,2^m-1$. Therefore, they generate the same point sets.
\end{pf*} 

%\begin{remark}
%For any $j \in \{1,\dots,s\}$, using two different generating matrices $C^{m}_j$ and $C'^{m}_j$ in equation (\ref{dig.net.eq.}) produce the same set of $2^m$ values $\{x_{0,j},\dots,x_{2^m-1,j}\}$ but ordered differently. It is possible to go from an order to the other by using the corresponding full rank upper triangular matrix $U^{m}_j$, such that $C^{m}_j=C'^{m}U^{m}_j$. This matrix can be thought as a permutation of the input binary vectors $\vi$.
%\end{remark}

\section{Iterative construction of replicated point sets}

We propose here two approaches to construct two replicated point sets $\mathcal{P}$ and $\mathcal{P}'$ based on Sobol' sequences. These two constructions are iterative that is the two points sets constructed possess the following structure:
$$\mathcal{P}= B_1 \cup B_2 \cup B_3 \dots $$
$$\mathcal{P}'= {B'}_1 \cup {B'}_2 \cup {B'}_3 \dots $$
where each $B_l$ (resp. $B'_l$), $l\geq1$, is a new set of points added at step $l$ to refine $\mathcal{P}$ (resp. $\mathcal{P}'$). At each step $l$ and for each approach, $\mathcal{P}$ and $\mathcal{P'}$ are two replicated designs of order $1$.

The first approach is called multiplicative approach and simply consists in generating two different Sobol' sequences. This approach exploits the result of Lemma \ref{Sobol_replicated}. The second approach is called additive approach. It takes an initial number of points of a Sobol' sequence and applies to it digital shift and scrambling operations to generate the two replicated points sets.


\subsection{Multiplicative approach}

The two replicated points sets $\mathcal{P}$ and $\mathcal{P}'$ constructed with the multiplicative approach are two $s$-dimensional Sobol' sequences constructed with different sets of generating matrices. We note $C_1,\dots,C_s$ the generating matrices used to generate design $\mathcal{P}$ and ${C'}_1,\dots,{C'}_s$ those used to generate design $\mathcal{P}'$. Our choice of the $2s$ generating matrices $C_1,\dots,C_S,{C'}_1,\dots,{C'}_s$ follows the results obtained by Joe and Kuo in \cite{kuo2}. These two authors proposed a set of generating matrices optimized based on the $t$-value of pairwise projections. We select $2s$ generating matrices of this set to generate our two designs $\mathcal{P}$ and $\mathcal{P}'$.
\bigskip

The iterative construction of designs $\mathcal{P}$ and $\mathcal{P}'$ proceed as follows:

at step $l$, $l \geq 1$, design $\mathcal{P}$ is refined with a set of $2^l$ points $B_l=\{\vx_{2^{l-1}},\dots,\vx_{2^l-1}\}$ obtained through equation (\ref{dig.net.eq.}) with $C_1^l,\dots,C_s^l$. Likewise, design $\mathcal{P}'$ is refined with ${B'}_l=\{\vx'_{2^{l-1}},\dots,\vx'_{2^l-1}\}$ obtained through equation (\ref{dig.net.eq.}) with ${C'}_1^l,\dots,{C'}_s^l$.
 
As a direct consequence of Lemma \ref{Sobol_replicated}, at each step $l$ the designs $\mathcal{P}$ and $\mathcal{P}'$ are two replicated designs of order $1$. Furthermore, they both inherit the space-filling properties of $(t,l,s)$-nets.


%The only constraint we have is that for each $u=1,\dots,s$, to estimate the quantity $\underline{\tau}_u^2$ the $2s$-dimensional Sobol' sequence must ensure that $x_{i,u}=x_{i,u+s}$ for $i\in\mathbb{N}_0$. Therefore, we will need $s$ different $2s$-dimensional sequences. These sequences can be constructed using a single $2s$-dimensional sequence and considering the corresponding upper triangular matrices $U_u$.

%Given the generators $C_1,\dots,C_s,C'_1,\dots,C'_s$, we define the $U_u$ matrices such that $C_u=C'_uU_u$. Then, each sequence generated by $C_1,\dots,C_s,C'_1U_u,\dots,C'_sU_u$ will satisfy $x_{i,u}=x_{i,u+s}$, for all $i\in\mathbb{N}_0$.

%\begin{proposition}
%For any $m\geq 0$, and $t_m$ depending on $m$, if the Sobol' $(t_m,m,s)$-nets are generated by $C_1,\dots,C_s$, then $C_1U,\dots,C_sU$, where $U$ is an upper triangular matrix, generate exactly the same $(t_m,m,s)$-nets.
%\end{proposition}
%\begin{pf*}{Proof.}
%For any $m$ and $2^{m-1}\leq i < 2^m$, the image by $U$, $\vj=U\vi$, will also satisfy $2^{m-1}\leq j < 2^m$. The reason is because $U$ is upper triangular, therefore $j_{m-1}=i_{m-1}$. Thus, by induction on $m$, the $(t_m,m,s)$-nets generated after applying the permutation $U$ contain exactly the same points.
%\end{pf*}

%This means that the use of these matrices $U_u$ do not affect the quality of our initial sequence.

%We still can optimize the quality of the $s$ sequences together by taking a different order of the generating matrices of the initial sequence. Given the generators $C_1,\dots,C_{2s}$, for all $s$ reordered sequences, the projections $C_i-C_j$ appear $s$ times when $0<i<j\leq s$, $s-1$ times when $0<i\leq s<j\leq 2s$, and $s-2$ times if $s<i<j\leq 2s$. Hence, we can sort the matrices such that sets of bad $C_{\pi(i)}-C_{\pi(j)}$ projections are mostly for $s<{\pi(i)}<{\pi(j)}\leq 2s$.

\subsection{Additive Approach}

The two replicated points sets $\mathcal{P}$ and $\mathcal{P}'$ constructed with the additive approach are iteratively refined with points sets $B_l$ and ${B'}_l$ that each possesses the structure of a $(t,m,s)$-net.

First, an initial set of $2^m$ points $B_0=\{\vx_i^{(0)}\}_{i=0}^{2^m-1}$ of a $s$-dimensional Sobol' sequence is generated by fixing $m >0$ and selecting $s$ generating matrices $C_1,\dots,C_s$ from the set obtained in \cite{kuo2}. Then, $B_l$ and ${B'}_l$ are obtained from $B_0$ by carrying out digital shift and scrambling operations. These operations guarantee that both $B_l$ and ${B'}_l$ inherit the $(t,m,s)$-net structure of the starting set $B_0$.
\bigskip

We detail below the construction of $B_l$ and ${B'}_l$:

At step $l$, $l \geq 1$, design $\mathcal{P}$ is refined with a set of $2^m$ points $B_l=\{\vx_i^{(l)}\}_{i=0}^{2^m-1}$. $B_l$ is a digitally shifted version of $B_0$:

for each $i=0,\dots,2^m-1$, the points $\vx_i^{(l)} = (x_{i,1}^{(l)},\dots,x_{i,s}^{(l)})^\intercal$ is obtained by:
\begin{equation}
\label{dig.shift}
(x_{i,j,1}^{(l)},\dots,x_{i,j,m}^{(l)})^\intercal = (x_{i,j,1}^{(0)},\dots,x_{i,j,m}^{(0)})^\intercal+\ve_j,\qquad j= 1,\dots,s\, ,
\end{equation}
where $ x_{i,j}^{(l)}=\sum_{k = 1}^mx_{i,j,k}^{(l)}2^{-k}$ and $\ve_j=(e_{j,{0}},\dots,e_{j,{m-1}})^\intercal \in ker(C_j^m)$ the kernel of $C_j^m$.
\bigskip

Then, design $\mathcal{P}'$ is refined with a set of $2^m$ points ${B'}_l=\{{\vx'}_i^{(l)}\}_{i=0}^{2^m-1}$. ${B'}_l$ is obtained from $B_l$ by applying Tezuka's i-binomial scrambling \cite{tezuka}. This scrambling operation writes as follows: 

for each $i=0,\dots,2^m-1$, the points ${\vx'}_i^{(l)} = ({x'}_{i,1}^{(l)},\dots,{x'}_{i,s}^{(l)})^\intercal$ is obtained by:
\begin{equation}
\label{ibinom.scrambling}
({x'}_{i,j,1}^{(l)},\dots,{x'}_{i,j,m}^{(l)})^\intercal = L_j^m (x_{i,j,1}^{(l)},\dots,x_{i,j,m}^{(l)})^\intercal,
\end{equation}
where $ {x'}_{i,j}^{(l)}=\sum_{k = 1}^m{x'}_{i,j,k}^{(l)}2^{-k}$ and $L_j^m$ is a full rank lower triangular matrix of size $m \times m$ over the Galois field $\mathbb{F}_2$.
\bigskip

%\item[Step 4.] The two blocks $B_l$ and ${B'}_l$ are then both randomized using the same Owen's scrambling column-wise.

%\item[Step 5.] $B_l$ and ${B'}_l$ are used to estimate all first-order indices $\underline{\tau}_u^2, u \in \mathcal{D}$.
%There is a finite choice for the vector $\ve_j$ and matrix $L_j^m$ respectively constructed in Steps $2$ and $3$. The choice of $m$ must take into account this limitation.

\begin{lemma}
At each step $l$, $l\geq1$, the digital shift and scrambling operations carried out in equations (\ref{dig.shift}) and (\ref{ibinom.scrambling}) ensure that $\mathcal{P}$ and $\mathcal{P}'$ are two replicated designs of order $1$. 
\end{lemma}

\begin{pf*}{Proof.}
This is a direct consequence of the following result: at each step $l$, $B_l$ and ${B'}_l$ are two replicated designs of order $1$. Indeed, since digital shift and i-binomial scrambling operations are bijections from $\mathbb{F}_2^m$ to $\mathbb{F}_2^m$, each dimension of $B_l$ and ${B'}_l$ contains the same set of $2^m$ values.
\end{pf*}

In the multiplicative approach, at each step $l$, $l \geq 1$ the sizes of designs $\mathcal{P}$ and $\mathcal{P}'$ are multiplied by $2$. This relatively fast growing size may be inadequate for some applications. The additive approach presented here is attractive due to the slow growing size of the two replicated designs $\mathcal{P}$ and $\mathcal{P}'$. At each step $l$, only $2^m$ points are added to both designs. However, the main drawback is that $\mathcal{P}$ and $\mathcal{P}'$ each as a whole does not possess the structure of a Sobol' sequence anymore. 

%In addition to it, one can still randomize the initial Sobol' sequence using Owen's scrambling \cite{owen.scrambl}. The only requirement to keep the replication property is to apply the same scrambling to dimensions that differ by $s$ positions, i.e., same scrambling to $C_j$ and $C_{j+s}$ for all $j=1,\dots,s$.

%\label{}
% etc, etc

% The Appendices part is started with the command \appendix;
% appendix sections are then done as normal sections
% \appendix

% \section{}
% \label{}

% The Acknowledgements are an un-numbered section
%\section*{Acknowledgements}
% Acknowledgements text here


\begin{thebibliography}{00}
% please try to use the bibitem system -
% the references should be in alphabetical order of authors' names.
% Articles with a single author first, author will 1 co-author next,
% then author with several co-authors;


% \bibitem{label}
% Text of bibliographic item

\bibitem{label}

\end{thebibliography}

\end{document}
