% Template article for preprint document class `elsart'
% SP 2001/01/05
% Modified CG (ESME) for Model 3, single column, 2 titles, abstract/r�sum�,
%  and 2 sets of keywords - 07.01.03 - file called Maths-English.tex
% English Version for Mathematics (CRAS series 1)
% Revamped, CG, 17.08.04, adding header, dates, and presenter

\documentclass{elsart3-1}

% Use the option doublespacing or reviewcopy to obtain double line spacing
% \documentclass[doublespacing]{elsart}

% if you use PostScript figures in your article
% use the graphics package for simple commands
% \usepackage{graphics}
% or use the graphicx package for more complicated commands
% \usepackage{graphicx}
% or use the epsfig package if you prefer to use the old commands
% \usepackage{epsfig}

% The amssymb package provides various useful mathematical symbols
\usepackage{amsmath,amssymb}

\usepackage[english,francais]{babel}

%ENVIRONMENTS THEOREMS...
% These are predefined, and follow the numbering system used in the journal!
%English
\newtheorem{theorem}{Theorem}[section]
\newtheorem{lemma}[theorem]{Lemma}
\newtheorem{e-proposition}[theorem]{Proposition}
\newtheorem{corollary}[theorem]{Corollary}
\newtheorem{e-definition}[theorem]{Definition\rm}
\newtheorem{remark}{\it Remark\/}
\newtheorem{example}{\it Example\/}
%French
\newtheorem{theoreme}{Th\'eor\`eme}[section]
\newtheorem{lemme}[theoreme]{Lemme}
\newtheorem{proposition}[theoreme]{Proposition}
\newtheorem{corollaire}[theoreme]{Corollaire}
\newtheorem{definition}[theoreme]{D\'efinition\rm}
\newtheorem{remarque}{\it Remarque}
\newtheorem{exemple}{\it Exemple\/}
\renewcommand{\theequation}{\arabic{equation}}
\setcounter{equation}{0}

\newcommand{\fudge}{\fC}
\newcommand{\dtf}{\textit{\doubletilde{f}}}
\newcommand{\cube}{[0,1)^s}
\newcommand{\rf}{\mathring{f}}
\newcommand{\rnu}{\mathring{\nu}}
\newcommand{\natm}{\naturals_{0,m}}
\newcommand{\wcS}{\widecheck{S}}
\newcommand{\tol}{\text{tol}}
\newcommand{\bvec}[1]{\boldsymbol{#1}}
\newcommand{\vx}{\bvec{x}}
\newcommand{\vi}{\bvec{i}}
\newcommand{\vj}{\bvec{j}}
\newcommand{\ve}{\bvec{e}}
\newcommand{\vk}{\bvec{k}}
\newcommand{\vz}{\bvec{z}}
\newcommand{\dif}{\mathsf{d}}
\newcommand{\hf}{\hat{f}}
\newcommand{\hS}{\widehat{S}}
\newcommand{\tS}{\widetilde{S}}
\newcommand{\tf}{\tilde{f}}
\newcommand{\fC}{\mathfrak{C}}
\newcommand{\homega}{\widehat{\omega}}
\newcommand{\wcomega}{\mathring{\omega}}
\newcommand{\vzero}{\bvec{0}}
\newcommand{\integers}{\mathbb{Z}}
\newcommand{\naturals}{\mathbb{N}}
\newcommand{\ip}[3][{}]{\ensuremath{\left \langle #2, #3 \right \rangle_{#1}}}

\def\abs#1{\ensuremath{\left \lvert #1 \right \rvert}}

%%%%%%%%%%%%%%%%%%%%%%%%%%%%%%%%
%% GUILLEMETS (FRENCH QUOTES) %%
%%%%%%%%%%%%%%%%%%%%%%%%%%%%%%%%
\def\og{\leavevmode\raise.3ex\hbox{$\scriptscriptstyle\langle\!\langle$~}}
\def\fg{\leavevmode\raise.3ex\hbox{~$\!\scriptscriptstyle\,\rangle\!\rangle$}}


\journal{the Acad\'emie des sciences}
\begin{document}
% place in the next line the header (rubrique) chosen for your article,
% if you know it (you can also have 2, format : Header1/Header2
\centerline{}
\begin{frontmatter}

% Title, authors and addresses

% use the thanksref command within \title, \author or \address for footnotes;
% use the ead command for the email address,
% and the form \ead[url] for the home page:
% \title{Title\thanksref{label1}}
% \thanks[label1]{}
% \author{Name\thanksref{label2}}
% \ead{email address}
% \ead[url]{home page}
% \thanks[label2]{}
% \address{Address\thanksref{label3}}
% \thanks[label3]{}
\selectlanguage{english}
\title{On the replication of Sobol' sequences to estimate main effects of model inputs}


% use optional labels to link authors explicitly to addresses:
% \author[label1,label2]{}
% \address[label1]{}
% \address[label2]{}
% The [label1] can be suppressed if there is only one address for all authors

\selectlanguage{english}
\author[authorlabel1]{Laurent Gilquin},
\ead{laurent.gilquin@inria.fr}
\author[authorlabel2]{Llu\'{i}s Antoni Jim\'{e}nez Rugama},
%\ead{ljimene1@hawk.iit.edu}
\author[authorlabel3]{Elise Arnaud},
\author[authorlabel2]{Fred J. Hickernell},
\author[authorlabel4]{Herv\'{e} Monod},
\author[authorlabel3]{Cl\'{e}mentine Prieur}

\address[authorlabel1]{Inria Grenoble - Rh\^{o}ne-Alpes, Inovall\'{e}e, 655 avenue de l'Europe, 38330 Montbonnot}
\address[authorlabel2]{TO FILL}
\address[authorlabel3]{Univ. Grenoble Alpes, Jean Kunzmann Laboratory, F-38000 Grenoble, France \\
CNRS, LJK, F-38000 Grenoble, France, Inria}
\address[authorlabel4]{MaIAGE, INRA, Universit\'{e} Paris-Saclay, 78350 Jouy-En-Josas, France}

% If you know the dates of reception, and acceptation you can put them now;
%  idem the name of the person presenting the Note

\medskip
\begin{center}
{\small Received *****; accepted after revision +++++\\
Presented by �����}
\end{center}

\begin{abstract}
\selectlanguage{english}
% Text of abstract in English
Your abstract in English here.
{\it To cite this article: A.
Name1, A. Name2, C. R. Acad. Sci. Paris, Ser. I 340 (2005).}

\vskip 0.5\baselineskip

\selectlanguage{francais}
% Text of abstract in French
\noindent{\bf R\'esum\'e} \vskip 0.5\baselineskip \noindent
{\bf Here is the title in French. }
Your resume in French here.
{\it Pour citer cet article~: A. Name1, A. Name2, C. R. Acad. Sci.
Paris, Ser. I 340 (2005).}

\end{abstract}
\end{frontmatter}

\selectlanguage{english}

\section{Introduction}
Mathematical models used in various fields often involve a substantial number of poorly known parameters. The effect of these parameters on the output of the model can be assessed through sensitivity analysis. Global sensitivity analysis methods are useful tools to identify the parameters having the most influence on the output. A well known approach is the variance based method introduced by Sobol' \cite{sobol'}. This method estimates sensitivity indices called Sobol' indices that summarize the influence of each model's input. In particular, one can distinguish first-order indices that estimate the main effect of each input.

The procedure proposed by Sobol' to estimate first-order Sobol' indices and its improvements (see Saltelli \cite{Saltelli} for an exhaustive survey) all suffer from a prohibitive cost of model evaluations that grows with respect to the input space dimension. %An elegant solution to reduce this cost relies on the construction of replicated designs introduced by McKay \cite{mckay}.  
 An elegant solution to reduce this cost relies on the construction of particular designs of experiments called replicated designs. The notion of replicated designs was first introduced by McKay through its introduction of replicated Latin Hypercubes \cite{mckay}. 
Here, replicated designs are defined in a wider framework:

\begin{definition}%[Replicated designs]
\label{rep.designs}
Denote by $\vx \in [0,1)^s$ a point with components $x_1,\dots,x_s$. Set $\mathcal{D}=\{1,\dots,s\}$. Let $\mathcal{P}=\{\vx_i\}_{i=0}^{n-1}$ and $\mathcal{P}'=\{{\vx'}_i\}_{i=0}^{n-1}$ be two point sets,
$\vx_i,\vx'_i\in[0,1)^{s}$. Let $\mathcal{P}^u=\{\vx_{i,u}\}_{i=0}^{n-1}$ (resp. $\mathcal{P}'^u$), $u \subsetneq \mathcal{D}$, denote the subset of dimensions of $\mathcal{P}$ (resp. $\mathcal{P}'$) indexed by $u$. We say that $\mathcal{P}$ and $\mathcal{P}'$ are two replicated designs of order $r$ if:\\
$\forall \ u \subsetneq \mathcal{D}$ such that $|u|=r$, $\mathcal{P}^u$ and $\mathcal{P}'^u$ are the same point set in $[0,1)^r$.
\end{definition}

With the construction of only two replicated designs of order $1$, the replication procedure introduced and studied in \cite{mara,tissot} allows to estimate all first-order Sobol' indices. This procedure has the major advantage of reducing drastically the estimation cost as the number of runs (one design $\mathcal{P}$ of size $n$  and its replication $\mathcal{P}'$) becomes independent of the input space dimension. However, if the input space is not properly explored the Sobol' indices estimates may still not be accurate enough. 
\bigskip

In this note, we propose to construct replicated point sets based on Sobol' sequences to generate two space-filling replicated designs of order $1$. These two designs insure that the input space is properly explored and can then be used within the replication procedure to estimate all main-effects of a numerical model. We first provide backgrounds on digital sequences, then we present two iterative approaches to construct replicated point sets based on Sobol' sequences. 

\section{Digital sequences background}

\subsection{Preliminaries}

The introduction of digital nets and sequences has its origin in numerical integration and uniform distribution. Explicit constructions of digital sequences have been developed notably by Sobol' and Niederreiter \cite{niderreiter} into the overarching notion of $(t,m,s)$-nets and $(t,s)$-sequences. A comprehensive theory of $(t,m,s)$-net and $(t,s)$-sequences was first formulate by Niederreiter. 

$(t,m,s)$-nets are defined as point sets in $\cube$ whose quality is measured by the parameter $t$, called $t$-value. They are defined as follows:
\begin{definition}%[$(t,m,s)-net$]
Let $\mathcal{A}$ be the set of all elementary intervals $A\in\cube$ where $A=\prod_{j=1}^s [\alpha_jb^{-\gamma_j},(\alpha_j+1)b^{-\gamma_j})$, with integers $s\geq 1$, $b\geq 2$, $\gamma_j\geq 0$, and $b^{\gamma_j}>\alpha_j\geq 0$. For $m\geq t\geq 0$, the point set $\mathcal{P}\in\cube$ with $b^m$ points is a $(t,m,s)-net$ in base $b$ if every $A$ with volume $b^{t-m}$ contains $b^t$ points of $\mathcal{P}$.
\end{definition}

A $(t,m,s)$-net is defined such that all elementary intervals of volume at least $b^{t-m}$ will enclose a proportional number of points. The most evenly spread nets are $(0,m,s)$-nets, since each elementary interval of the smallest volume possible, $b^{-m}$, contains exactly one point. %Note that if $t'>t$, $(t,m,s)$-nets are always $(t',m,s)$-nets.

$(t,s)$-sequences are the infinite analogues of $(t,m,s)$-nets. They are defined as follows:
\begin{definition}%[$(t,s)$-sequence]
For integers $s\geq 1$, $b\geq 2$, and $t\geq 0$, the sequence $\{\vx_i\}_{i\in\mathbb{N}_0}$ is a $(t,s)$-sequence in base $b$, if for every set $\mathcal{P}_{\ell,m}=\{\vx_i\}_{i=\ell b^m}^{(\ell+1)b^m-1}$ with $\ell\geq 0$ and $m\geq t$, $\mathcal{P}_{\ell,m}$ is a $(t,m,s)$-net in base $b$.
\end{definition}
\bigskip

All $(t,m,s)$-nets and $(t,s)$-sequences relevant for applications are based on a digital construction scheme over a finite field $\mathbb{F}_b$ with $b$ a prime power number. Digital nets and sequences referred to these structures. 
\bigskip

The construction scheme of a $s$-dimensional digital sequence requires the knowledge of the base $b$ and a set of $s$ full rank upper triangular matrices of infinite size called generating matrices. These generating matrices are recursively constructed through primitive polynomials and initial directional numbers (see \cite{niederreiter} for further details). Denote by $C_1,\dots,C_s$ $s$ generating matrices over $\mathbb{F}_b$. For $m>0$ and $j \in \{1,\dots,s\}$, we note $C_j^{m}$ the $m \times m$ upper left block of $C_j$. The first $b^m$ points $\{\vx_0,\dots,\vx_{b^m-1}\}$ of the digital sequence are constructed as follows:\\

For each $i=0,\dots,b^m-1$, the point $\vx_i = (x_{i,1},\dots,x_{i,s})^\intercal$ of the sequence is obtained dimension-wise by:
\begin{equation}
\label{dig.net.eq.}
(x_{i,j,1},x_{i,j,2},\dots)^\intercal = C_j^{m} \vi,\qquad j= 1,\dots,s\, ,
\end{equation}
where $x_{i,j} = \sum_{k \geq 1}x_{i,j,k}b^{-k}$ and $\vi = (i_{0},\dots,i_{m-1})^\intercal$ is the b-ary decomposition of $i$.
\bigskip

For $m' > m$ the matrix $C_j^m$ is embedded in the upper left of matrices $C_j^{m'}$. Resulting from this embedding, the construction of a digital sequence can be performed iteratively. To link digital sequences to the definition of replicated designs, we introduce the following concept:
\begin{definition}
Two digital sequences $\{\vx_i\}_{i\in\mathbb{N}_0}$ and $\{{\vx'}_i\}_{i\in\mathbb{N}_0}$ are said \emph{digitally replicated} of order $r$ if for all $m\geq 0$, $\{{\vx}_i\}_{i=0}^{b^m-1}$ and $\{{\vx'}_i\}_{i=0}^{b^m-1}$ are two replicated designs of order $r$.
\end{definition}
 
\subsection{Sobol' sequences}

Sobol' sequences are digital sequences constructed in base $2$. These sequences are attractive for their easy implementation and optimized properties \cite{kuo2}. Sobol' sequences are particularly interesting for having full rank upper triangular generating matrices over $\mathbb{F}_2$. This property leads to the following lemma:
\begin{lemma}\label{Sobol_replicated}
All Sobol' sequences are digitally replicated of order 1.
\end{lemma}
\begin{pf*}{\textit{Proof}.}
Consider any two s-dimensional Sobol' sequences generated respectively by the matrices $C_1,\dots,C_s$ and $C'_1,\dots,C'_s$. Since these matrices are upper triangular, to compute the first $2^m$ points one only needs the knowledge of the $m\times m$ upper left blocks $C_j^m$ and ${C'}_j^m$, $j \in \{1,\dots,s\}$. Because any two matrices $C^{m}_j$ and ${C'}^{m}_j$ are square and full rank, the products $C^{m}_j\vi$ and ${C'}^{m}_j\vi$ are one-to one and onto, for all $i=0,\dots,2^m-1$. Therefore, they generate the same point sets.
\end{pf*} 
For $j \in \{1,\dots,s\}$, using two different generating matrices $C^{m}_j$ and $C'^{m}_j$ in equation (\ref{dig.net.eq.}) produce the same set of $2^m$ values $\{x_{0,j},\dots,x_{2^m-1,j}\}$ but ordered differently. It is possible to go from an order to the other by using the corresponding full rank upper triangular matrix $U^{m}_j$, such that $C^{m}_j=C'^{m}U^{m}_j$. This matrix can be thought as a permutation of the input binary vectors $\vi$.


\section{Iterative construction of replicated point sets}

%In section \ref{sobol.seq.cons}, we present two approaches to construct replicated point sets based on Sobol' sequences. The multiplicative approach uses two Sobol' sequences that are replicated by construction, while the additive approach takes an initial number of points of two Sobol' sequences and applies digital shifts and scramblings to extend the point set. In this case, at step $k$ the size of each replicated Sobol' sequence is $ k \times 2^m$ where $2^m$ is the size of the initial Sobol' sequence. 

\subsection{Multiplicative approach}
Denote by $\mathcal{P}$ and $\mathcal{P}'$ the two designs iteratively refined with new points sets using the multiplicative approach. %Based on Lemma \ref{Sobol_replicated}, by taking $\mathcal{P}$ and $\mathcal{P}'$ to be two $s$-dimensional Sobol' sequences we obtain two replicated designs or order $1$ with an innate nested structure.
The goal of this approach is to generate a $2s$-dimensional Sobol' sequence with good properties to estimate all first order Sobol' indices. Design $\mathcal{P}$ is formed with the first $s$ dimensions of the sequence and design $\mathcal{P}'$ with the $s$ remaining ones. As shown in Lemma \ref{Sobol_replicated}, for any $m\geq 0$ and $2^m$ points, each dimension of the sequence will have exactly the same values, although probably with different order. Thus, designs $\mathcal{P}$ and $\mathcal{P}'$ are nested replicated designs of order $1$ and both are $s$-dimensional Sobol' sequences.

The only constraint we have is that for each $u=1,\dots,s$, to estimate the quantity $\underline{\tau}_u^2$ the $2s$-dimensional Sobol' sequence must ensure that $x_{i,u}=x_{i,u+s}$ for $i\in\mathbb{N}_0$. Therefore, we will need $s$ different $2s$-dimensional sequences. These sequences can be constructed using a single $2s$-dimensional sequence and considering the corresponding upper triangular matrices $U_u$.

Given the generators $C_1,\dots,C_s,C'_1,\dots,C'_s$, we define the $U_u$ matrices such that $C_u=C'_uU_u$. Then, each sequence generated by $C_1,\dots,C_s,C'_1U_u,\dots,C'_sU_u$ will satisfy $x_{i,u}=x_{i,u+s}$, for all $i\in\mathbb{N}_0$.

\begin{proposition}
For any $m\geq 0$, and $t_m$ depending on $m$, if the Sobol' $(t_m,m,s)$-nets are generated by $C_1,\dots,C_s$, then $C_1U,\dots,C_sU$, where $U$ is an upper triangular matrix, generate exactly the same $(t_m,m,s)$-nets.
\end{proposition}
\begin{pf*}{Proof.}
For any $m$ and $2^{m-1}\leq i < 2^m$, the image by $U$, $\vj=U\vi$, will also satisfy $2^{m-1}\leq j < 2^m$. The reason is because $U$ is upper triangular, therefore $j_{m-1}=i_{m-1}$. Thus, by induction on $m$, the $(t_m,m,s)$-nets generated after applying the permutation $U$ contain exactly the same points.
\end{pf*}

This means that the use of these matrices $U_u$ do not affect the quality of our initial sequence.

Sobol' generating matrices $C$ are constructed through primitive polynomials and initial directional numbers. How to choose them is deeply discussed in \cite{kuo}. The $t$-values of pairwise projections are already optimized in \cite{kuo2}. Thus, we will consider the $2s$-dimensional Sobol' sequence described by them to construct the other $s$ $2s$-dimensional sequences.

We still can optimize the quality of the $s$ sequences together by taking a different order of the generating matrices of the initial sequence. Given the generators $C_1,\dots,C_{2s}$, for all $s$ reordered sequences, the projections $C_i-C_j$ appear $s$ times when $0<i<j\leq s$, $s-1$ times when $0<i\leq s<j\leq 2s$, and $s-2$ times if $s<i<j\leq 2s$. Hence, we can sort the matrices such that sets of bad $C_{\pi(i)}-C_{\pi(j)}$ projections are mostly for $s<{\pi(i)}<{\pi(j)}\leq 2s$.

In addition to it, one can still randomize the initial Sobol' sequence using Owen's scrambling \cite{owen.scrambl}. The only requirement to keep the replication property is to apply the same scrambling to dimensions that differ by $s$ positions, i.e., same scrambling to $C_j$ and $C_{j+s}$ for all $j=1,\dots,s$.

\subsection{Additive Approach}

Denote by $\mathcal{P}$ and $\mathcal{P}'$ the two replicated designs iteratively refined with new points sets using the additive approach. These two designs possess a block structure:
$$\mathcal{P}= B_0 \cup B_1 \cup B_2 \dots $$
$$\mathcal{P}'= {B'}_0 \cup {B'}_1 \cup {B'}_2 \dots $$
We detail below the steps of the construction of these two designs.
\begin{enumerate}
\item[Step 1.] We start by fixing $m$ and choosing a set $C_1^{m},\dots,C_s^{m}$ of generating matrices to construct an initial Sobol' sequence $\{\vx_i\}_{i=0}^{2^m-1}$ of $2^m$ points. \\

\item[Step 2.] At step $l$ of the recursive estimation, $l \geq 0$, a new points set $B_l=\{\vx_i^{(l)}\}_{i=0}^{2^m-1}$ of $2^m$ points is added to $\mathcal{P}$. $B_l$ is obtained by applying a digital shift to the initial Sobol' sequence. The digital shift operation is applied as follows:

for each $i=0,\dots,2^m-1$, the points $\vx_i^{(l)} = (x_{i,1}^{(l)},\dots,x_{i,s}^{(l)})^\intercal$ is obtained by:
\begin{equation}
\label{dig.shift}
(x_{i,j,1}^{(l)},\dots,x_{i,j,m}^{(l)})^\intercal = C_j^m \vi+\ve_j,\qquad j= 1,\dots,s\, ,
\end{equation}
where $ x_{i,j}^{(l)}=\sum_{k = 1}^mx_{i,j,k}^{(l)}2^{-k}$ and $\ve_j=(e_{j,{0}},\dots,e_{j,{m-1}})^\intercal \in ker(C_j^m)$ the kernel of $C_j^m$.\\

\item[Step 3.] Then, the points set ${B'}_l=\{{\vx'}_i^{(l)}\}_{i=0}^{2^m-1}$ is obtained from $B_l$ by applying Tezuka's i-binomial scrambling \cite{tezuka}. This operation consists in the following: 

for each $i=0,\dots,2^m-1$, the points ${\vx'}_i^{(l)} = ({x'}_{i,1}^{(l)},\dots,{x'}_{i,s}^{(l)})^\intercal$ is obtained by:
\begin{equation}
\label{ibinom.scrambling}
({x'}_{i,j,1}^{(l)},\dots,{x'}_{i,j,m}^{(l)})^\intercal = L_j^m (x_{i,j,1}^{(l)},\dots,x_{i,j,m}^{(l)})^\intercal,
\end{equation}
where $ {x'}_{i,j}^{(l)}=\sum_{k = 1}^m{x'}_{i,j,k}^{(l)}2^{-k}$ and $L_j^m$ is an invertible lower triangular matrix of size $m \times m$ over the Galois field $\mathbb{F}_2$.\\

\item[Step 4.] The two blocks $B_l$ and ${B'}_l$ are then both randomized using the same Owen's scrambling column-wise.

\item[Step 5.] $B_l$ and ${B'}_l$ are used to estimate all first-order indices $\underline{\tau}_u^2, u \in \mathcal{D}$.
\end{enumerate}
Steps $2$ to $5$ are iterated until a stopping criterion is reached. There is a finite choice for the vector $\ve_j$ and matrix $L_j^m$ respectively constructed in Steps $2$ and $3$. %Once $m$ is fixed, one can build up to $2^m$ different $\ve_j$ and $2^{m(m-1)/2}$ different $L_j^m$. 
The choice of $m$ must take into account this limitation.

 At each step $l$, $B_l$ and ${B'}_l$ are two replicated designs of order $1$ thus enabling the estimation of all first-order indices. Denote by $K$ the step at which the recursive estimation stopped. The total cost of the additive approach to estimate all first-order indices equals $2 \times K \times 2^m$. 
 
The additive approach is attractive due to the slow growing size of the two replicated designs $\mathcal{P}$ and $\mathcal{P}'$. At the opposite of the multiplicative approach, the main drawback is that $\mathcal{P}$ and $\mathcal{P}'$ does not possess the structure of a Sobol' sequence but each block $B_l$ (resp. $B'_l$) composing them does.


%\label{}
% etc, etc

% The Appendices part is started with the command \appendix;
% appendix sections are then done as normal sections
% \appendix

% \section{}
% \label{}

% The Acknowledgements are an un-numbered section
%\section*{Acknowledgements}
% Acknowledgements text here


\begin{thebibliography}{00}
% please try to use the bibitem system -
% the references should be in alphabetical order of authors' names.
% Articles with a single author first, author will 1 co-author next,
% then author with several co-authors;


% \bibitem{label}
% Text of bibliographic item

\bibitem{label}

\end{thebibliography}

\end{document}
