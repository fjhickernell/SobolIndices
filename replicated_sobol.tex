% !TEX TS-program = PDFLatexBibtex
%&LaTeX
\documentclass[]{elsarticle}
\setlength{\marginparwidth}{0.5in}
\usepackage{amsmath,amssymb,amsthm,natbib,mathtools,bbm,extraipa,mathabx,graphicx}
\usepackage[ruled,vlined]{algorithm2e}
%accents,

\newtheorem{lem}{Lemma}
\newtheorem{remark}{Remark}
\theoremstyle{definition}
\newtheorem{defin}{Definition}
\newtheorem{prop}{Proposition}
\newtheorem{algo}{Algorithm}

\newcommand{\fudge}{\fC}
\newcommand{\dtf}{\textit{\doubletilde{f}}}
\newcommand{\cube}{[0,1)^s}
%\renewcommand{\bbK}{\natzero^d}
\newcommand{\rf}{\mathring{f}}
\newcommand{\rnu}{\mathring{\nu}}
\newcommand{\natm}{\naturals_{0,m}}
\newcommand{\wcS}{\widecheck{S}}
\newcommand{\tol}{\text{tol}}
\newcommand{\e}{\text{e}}
\newcommand{\bvec}[1]{\boldsymbol{#1}}
\newcommand{\vx}{\bvec{x}}
\newcommand{\vi}{\bvec{i}}
\newcommand{\ve}{\bvec{e}}
\newcommand{\vk}{\bvec{k}}
\newcommand{\vz}{\bvec{z}}
\newcommand{\dif}{\mathsf{d}}
\newcommand{\hf}{\hat{f}}
\newcommand{\hS}{\widehat{S}}
\newcommand{\tS}{\widetilde{S}}
\newcommand{\tf}{\tilde{f}}
\newcommand{\fC}{\mathfrak{C}}
\newcommand{\homega}{\widehat{\omega}}
\newcommand{\wcomega}{\mathring{\omega}}
\newcommand{\vzero}{\bvec{0}}
\newcommand{\integers}{\mathbb{Z}}
\newcommand{\naturals}{\mathbb{N}}
\newcommand{\ip}[3][{}]{\ensuremath{\left \langle #2, #3 \right \rangle_{#1}}}

\def\abs#1{\ensuremath{\left \lvert #1 \right \rvert}}

\begin{document}

\begin{frontmatter}

\title{}

\author{Cl\'ementine Prieur, Elise Arnaud, Herv\'{e} Monod, Laurent Gilquin, Fred J. Hickernell, Llu\'{i}s Antoni Jim\'{e}nez Rugama}
\address{U. Josef Fourier, Illinois Institute of Technology}
\begin{abstract}
\end{abstract}

\end{frontmatter}

\section{Problem Statement}
Denote by $\vx \in [0,1)^s$ a point with components $x_1,\dots,x_s$. We set $\mathcal{D}=\{1,\dots,s\}$. Let $u$ be a subset of $\mathcal{D}$, $-u$ its complement and $|u|$ its cardinality. $\vx_u$ represents a point in $[0,1)^u$ with components $x_j, j \in u$. Given two points $\vx$ and $\vx'$, we define the following hybrid point: 
\[\vz_u=(\vx_u:{\vx'}_{-u}), \text{ where: } \left\{
      \begin{aligned}
        z_j &= x_j, & \text{ if } j \in u\\
        z_j &= {x'}_j, & \text{ if } j \in -u\\
      \end{aligned}
    \right.  \]
Let $f$ be the symbol representing the numerical model considered. We assume $f \in \mathcal{L}^2[0,1]^s$. Denote by $\mu$ and $\sigma^2$ the mean and variance of $f$.
%These two quantities can be expressed as the following integrals:
%\[\mu= \frac{1}{2} \int_{[0,1)^{2d-1}}f(\vx)+f(\vx_u:{\vx'}_{-u}) d\vx d{\vx'}_{-u}, \]
%\[\sigma^2= \int_{[0,1)^{d}}f(\vx)^2 d\vx - \Big{(}\int_{[0,1)^{d}}f(\vx) d\vx\Big{)}^2 .\]
The uncertainty on $\vx$ is modeled by random variables such that $\vx \stackrel{iid}{\sim} \mathcal{U}[0,1)^s$. Recall the Hoeffding decomposition of $f$:
\begin{equation}
f(\vx)=\mu+\sum_{\substack{u \subseteq \{1,\dots,s\} \\ u \neq \emptyset}} f_u(\vx_u),
\label{anova}
\end{equation}
where: $\mu$ is a constant and $\forall \ u \subseteq \mathcal{D}, \ u \neq \emptyset, \ \forall \ j \in u: \int_{[0,1]} f_u(\vx_u) d{x_j}=0$.

Due to orthogonality, by taking the variance of each side in Equation \ref{anova} we obtain the following decomposition of the variance:
\[ \sigma^2 = \sum_{\substack{u \subseteq \{1,\dots,s\} \\ u \neq \emptyset}} \sigma_v^2.\]
The problem of interest is the estimation of the quantities:
\[\underline{\tau}_u^2 = \sum \limits_{v \subseteq u} \sigma_v^2, \qquad u \in \mathcal{D}.\]
These quantities correspond to the partial sensitivity Sobol' indices often found written in the literature in their normalized form: $\underline{\tau}_u^2/\sigma^2$. When, $|u|=1$, $\underline{\tau}_u^2$ corresponds to the first-order Sobol' index that evaluate the main effect of the variable $\vx_u$. Higher order indices evaluate higher order interactions between variables. In \cite{}, Owen proposed to use the following expression for $\underline{\tau}_u^2$:
\begin{equation}\label{first_order_indice}
\underline{\tau}_u^2  =\int_{[0,1)^{2s-1}}(f(\vx) - \mu) (f(\vx_u:{\vx'}_{-u})-\mu) d\vx d{\vx'}_{-u},
\end{equation}
taking $\mu$ as introduced in Janon \textit{et al.}:
\[\mu= \frac{1}{2} \int_{[0,1)^{2s-1}}f(\vx)+f(\vx_u:{\vx'}_{-u}) d\vx d{\vx'}_{-u}. \]
The natural way to estimate these two quantities is for $(\vx_i,\vx'_i) \stackrel{iid}{\sim} [0,1)^{2s}$ via:
\begin{align}
\widehat{\underline{\tau}_u^2} & = \frac{1}{n} \sum \limits_{i=1}^n (f(\vx_i) - \widehat{\mu}) (f(\vx_{i,u}:{\vx'}_{i,-u})-\widehat{\mu}),\label{first_order_indice_estimator}\\
\nonumber
\widehat{\mu} & = \frac{1}{2n} \sum \limits_{i=1}^n f(\vx_i) +f(\vx_{i,u}:{\vx'}_{i,-u}).
\end{align}

We consider here a quasi-Monte Carlo sampling strategy to evaluate these two estimators. We focus on the estimation of all first-order Sobol' indices, $u \in \mathcal{D}$. The classical estimation procedure introduced by Sobol' requires to evaluate the quantity $f(\vx_u:{\vx'}_{-u})$ $n$ times for each $u$ while the quantity $f(\vx)$ is only evaluated $n$ times once for all $u$. The total cost sums up to $n(d+1)$ evaluations of $f$. This cost can rapidly become prohibitive for models whose evaluation is time consuming, as such involve an important number of parameters. 

An improvement to reduce this cost lies in the use of replicated designs. The notion of replicated designs was first introduced by McKay through the introduction of replicated Latin Hypercubes samples. Here, we define replicated designs in a wider framework:
\begin{defin}[Replicated designs]
Let $\mathcal{P}=\{\vx_i\}_{i=0}^{n-1}$ and $\mathcal{P}'=\{{\vx'}_i\}_{i=0}^{n-1}$ be two point sets,
%\[\arraycolsep=1.3pt
%D=\left(\begin{array}{ccccc}
%x_{1,1} & ... & x_{1,j} & ... & x_{1,s} \\
%\vdots &  & \vdots & & \vdots \\
%x_{i,1} & ... & x_{i,j} & ... & x_{i,s} \\
%\vdots &  & \vdots & & \vdots \\
%x_{n,1} & ... & x_{n,j} & ... & x_{n,s} \\
%\end{array} \right), \ D'=\left(\begin{array}{ccccc}
%{x'}_{1,1} & ... & {x'}_{1,j} & ... & {x'}_{1,s} \\
%\vdots &  & \vdots & & \vdots \\
%{x'}_{i,1} & ... & {x'}_{i,j} & ... & {x'}_{i,s} \\
%\vdots &  & \vdots & & \vdots \\
%{x'}_{n,1} & ... & {x'}_{n,j} & ... & {x'}_{n,s} \\
%\end{array} \right).
%\]
$\vx_i,\vx'_i\in[0,1)^{s}$. Let $\mathcal{P}^u=\{\vx_{i,u}\}_{i=0}^{n-1}$ (resp. $\mathcal{P}'^u$), $u \subsetneq \mathcal{D}$, denote the subset of dimensions of $\mathcal{P}$ (resp. $\mathcal{P}'$) indexed by $u$. We say that $\mathcal{P}$ and $\mathcal{P}'$ are two replicated designs of order $r$ if:\\
$\forall \ u \subsetneq \mathcal{D}$ such that $|u|=r$, $\mathcal{P}^u$ and $\mathcal{P}'^u$ are the same point set in $[0,1)^r$.
\end{defin}
The replication procedure introduced in \cite{•} allows to estimate all first-order Sobol' indices with two replicated designs of order $1$. In this procedure, the quantity $f(\vx_u:{\vx'}_{-u})$ is only evaluated $n$ times once for all $u$, resulting in a total of $2n$ evaluations of $f$. %Denote by $D$,$D'$ the two replicated designs of order $1$ required by the replication procedure, $D'$ is constructed from $D$ as follows:
%\[{x'}_{i,j}=x_{\pi_j(i),j}, \qquad \forall \ (i,j) \in \{1,\dots,n\} \times \{1,\dots,d\},\]
%where $\pi_1,\dots,\pi_d$ are $d$ permutations on $\{1,\dots,n\}$ selected randomly and independently.

Our objective here is to combine the use of Sobol' sequences with the replication method to iteratively estimate all quantities $\underline{\tau}_u^2, u \in \mathcal{D}$, $\abs{u}=1$. This approach reduces to construct two replicated Sobol' sequences iteratively refined with new point sets. In this fashion, two approaches are considered: one multiplicative and one additive. In the next section, we provide some background on digital nets and sequences. The two approaches are detailed in Section \ref{sobol.seq.cons}.

\section{Digital Nets Background}

In this section we introduce a particular type of quasi-Monte Carlo points to estimate the quantities in $\eqref{first_order_indice}$ for $\abs{u}=1$, according to the replication method. 
%, which will allow us to evaluate the integrand at points $\{\left(\vx_i,{\vx'}_{i,-u}\right)\}_{i=0}^{b^m}$ only one time to estimate all $d$ indices.
The concepts of $(t,m,s)$-net and $(t,s)$-sequence were first introduced by Niederreiter in \cite{•}. Digital nets are accurately defined as $(t,m,s)$-nets in $\cube$ whose quality is measured by the parameter $t$, called $t$-value. %Because this parameter is the criterion we use to construct our new sequences, we provide the key definitions below.
%$(t,m,s)$-net and and $(t,s)$-sequence are strongly linked to the concept of star discrepancy \cite{.} and defined as follows:
\begin{defin}[$(t,m,s)-net$]
Let $\mathcal{A}$ be the set of all elementary intervals $A\in\cube$ where $A=\prod_{j=1}^s [\alpha_jb^{-\gamma_j},(\alpha_j+1)b^{-\gamma_j})$, with integers $s\geq 1$, $b\geq 2$, $\gamma_j\geq 0$, and $b^{\gamma_j}>\alpha_j\geq 0$. For $m\geq t\geq 0$, the point set $\mathcal{P}\in\cube$ with $b^m$ points is a $(t,m,s)-net$ in base $b$ if every $A$ with volume $b^{t-m}$ contains $b^t$ points of $\mathcal{P}$.
\end{defin}

A $(t,m,s)$-net is defined such that all elementary intervals of volume at least $b^{t-m}$ will enclose a proportional number of points. The most evenly spread nets are $(0,m,s)$-nets, since each elementary interval of the smallest size possible, $b^{-m}$, contains exactly one point. Note that if $t'>t$, $(t,m,s)$-nets are always $(t',m,s)$-nets.

\begin{defin}[$(t,s)$-sequence]
For integers $s\geq 1$, $b\geq 2$, and $t\geq 0$, the sequence $\{\vx_i\}_{i\in\mathbb{N}_0}$ is a $(t,s)$-sequence in base $b$, if every set $\mathcal{P}_{\ell,m}=\{\vx_i\}_{i=\ell b^m}^{(\ell+1)b^m-1}$ with $\ell\geq 0$ and $m\geq t$, $\mathcal{P}_{\ell,m}$ is a $(t,m,s)$-net in base $b$.
\end{defin}

Digital sequences are usually computed with $\infty\times\infty$-matrices $C_1,\dots,C_s$ over $\mathbb{F}_b$, often referred in the literature as generating matrices. However, to construct the first $b^m$ points, one only needs the knowledge of the blocks $C_1^{\infty\times m},\dots,C_s^{\infty\times m}$. This is due to its construction, described as follows. For each $i=0,\dots,b^m-1$, the point $\vx_i = (x_{i,1},\dots,x_{i,s})^\intercal$ of the sequence is obtained dimension per dimension:
\begin{equation}
\label{dig.net.eq.}
(x_{i,j,1},x_{i,j,2},\dots)^\intercal = C_j^{\infty\times m} (i_{0},\dots,i_{m-1})^\intercal,\qquad j= 1,\dots,s\, ,
\end{equation}
where $x_{i,j} = \sum_{k \geq 1}x_{i,j,k}b^{-k}$ and $i = \sum_{k=0}^{m-1}i_kb^{k}$. For $m' < m$ the matrices $C_j^{m'}$ are embedded in the upper left corners of matrices $C_j^{m}$. Thus, the digital nets possess an innate nested structure resulting from this embedding. That is $\mathcal{P}_{m'} \subset \mathcal{P}_m \subset \dots \subset \mathcal{P}_{\infty}$. 

Sobol' sequences are digital nets defined in base $b=2$. They are attractive due to their efficient implementation through the use of Gray code. In our scope, their attractiveness comes from their low growing rate (factor $2$). 
\bigskip

In the next section, we present two approaches to construct two replicated Sobol' sequences iteratively refined with new points sets. The multiplicative approach uses the innate nested structure of a Sobol' sequence. The size of each replicated Sobol' sequence is doubled at each step. The additive approach combines digital shifting and scrambling operations to form the new points set. In this case, at step $k$ the size of each replicated Sobol' sequence is $ k \times 2^m$ where $2^m$ is the size of the initial Sobol' sequence. 

\section{Iterative constructions of replicated Sobol' sequence}
\label{sobol.seq.cons}

\subsection{Replicated Sobol' sequences}


We start by introducing some useful definitions for the rest of the paper. We note by $\vi=(i_{0},\dots,i_{m-1})^T$ the vector resulting from the binary decomposition $\eqref{dig.net.eq.}$ of integer $i$. We also note by $U_m(\mathbb{F}_2)$ the multiplicative group of all invertible triangular matrices of size $m\times m$ over the Galois field $\mathbb{F}_2$. Consider the following application:
\begin{equation*}
 \Diamond \colon \left\lbrace \begin{array}{ccccc} (\mathbb{F}_2)^m \times U_m(\mathbb{F}_2) & \to & (\mathbb{F}_2)^m \\
  (\vi,C) & \mapsto &  C . \vi \\
\end{array} \right.
\label{model}
\end{equation*}
The application $\Diamond$ consists in the operation carried on in Equation $\eqref{dig.net.eq.}$. It is easy to see that for $C \in U_m(\mathbb{F}_2)$ fixed, $\Diamond$ is a bijection of $(\mathbb{F}_2)^m$. As a results, applying $\Diamond$ for all $\vi \in (\mathbb{F}_2)^m$ with any generating matrix $C_j^m$ give the same points set. This leads to the following definition of replicated Sobol' sequences:
\begin{defin}
Let $C_j^m$, $C_{j'}^m$ be two generating matrices. The corresponding Sobol' sequences $\{\vx_i\}_{i=0}^{b^m-1}$ and $\{\vx'_i\}_{i=0}^{b^m-1}$ are two replicated designs of order $1$. Furthermore, the digital sequences $\{{\vx}_i\}_{i=0}^{\infty}$ and $\{{\vx'}_i\}_{i=0}^{\infty}$ are two replicated Sobol' sequence, that is for all $m \geq 0$, $\{{\vx}_i\}_{i=b^{m-1}}^{b^m-1}$ and $\{{\vx'}_i\}_{i=b^{m-1}}^{b^m-1}$ are two replicated designs of order $1$.
\end{defin}
In other worlds, Sobol' sequences are innate replicated designs of order $1$ possessing a nested structure.


\subsection{Multiplicative approach}
%
%\begin{defin}
%A digital sequence $\{{\vx'}_i\}_{i\in\mathbb{N}_0}$ is the \emph{replicated digital} sequence of $\{\vx_i\}_{i\in\mathbb{N}_0}$ if for all $m\geq 0$, $\{{\vx'}_i\}_{i=0}^{b^m-1}$ is a replicated design of $\{{\vx}_i\}_{i=0}^{b^m-1}$.
%\end{defin}
%In this case, $C_j'$ will depend on the choice of $C_j$.

%If we do higher order Sobol' indices, we can easily show that $C'_{i}=AC_{i}$ for $i = 1,\dots,d$, and this does not work because this means having the same function values for each $u$, with a unique ordering given by $A$.
 
Although the choices of $C_j$ and $C_j'$ are many, there is a special case interesting for its construction simplicity, and wide range of $C_j'$ choices.

For this special case, we consider the generating matrices to be upper triangular. This ensures that $x_{i,j,k}=0$, for $m<k$ and $0 \leq i \leq b^m-1$. If we also assume that there are no zeros on the diagonal, the block $C_j^{m\times m}$ is full rank and each $x_{i,j}\neq x_{r,j}$ when $i\neq r$. Therefore, excluding the trivial case, it is enough to choose $C_j\neq C_j'$ for at least $d-1$ dimensions to obtain a \emph{replicated digital sequence}. 
%When $b=2$, these are Sobol' sequences.

For quasi-Monte Carlo integration purposes, given $\{{\vx}_i\}_{i\in\mathbb{N}_0}$ generated by $C_{1},\dots,C_{d}$, and $\{{\vx'}_i\}_{i\in\mathbb{N}_0}$ generated by $C'_{1},\dots,C'_{d}$, the $d$ sequences in dimension $2d-1$ used to estimate all $\underline{\tau}_u^2$ are generated by
\begin{equation}\label{ordered_sequence}
\underbrace{C_{1},\dots,C_{d}}_{\text{First }d\text{ dimensions}},\underbrace{C_{\alpha(u,1)},\dots,C_{\alpha(u,u-1)},C_{\alpha(u,u+1)},\dots,C_{\alpha(u,d)}}_{\text{Next }d-1\text{ dimensions}},\quad 1\leq u \leq d\, ,
\end{equation}
where $C_{\alpha(u,i)}:=C_u(C'_{u})^{-1}C'_{i}$. Note that for $i=u$, one has $C_{\alpha(u,u)}=C_u$. This is in fact forcing $C_u = C'_{u}$, as desired to apply the replicated method. Therefore, our goal is to find a good choice of $C_{1},\dots,C_{d},C'_{1},\dots,C'_{d}$ such that for all $u=1,\dots,d$, the sequence used to estimate $\underline{\tau}_u^2$ has a low $t$-value.

The best choice of primitive polynomials and directional numbers to generate good quality Sobol' sequences is deeply discussed in \cite{•}. We are going to take Kuo and Joe results to find  good \emph{replicated Sobol'} sequences.

Pairwise projections are already well optimized (see\cite[Table 2.1]{•}). Thus, our starting point will be using the $C_1,\dots,C_d$ found in \cite{•} to generate the sequence $\{{\vx}_i\}_{i\in\mathbb{N}_0}$. Hence, we will focus on finding the optimal $C'_{1},\dots,C'_{d}$ such that we minimize a $t$-value criteria (to specify, perhaps the max of all) over all $d$ sequences described in \eqref{ordered_sequence}.

%\subsection{Matrix generators group $U_m$}
%
%Be $U_m(\mathbb{F}_b)$ the multiplicative group of all invertible triangular matrices of size $m\times m$ over $\mathbb{F}_b$ ($GF(b)$). We have that $\abs{U_m(\mathbb{F}_b)}=(b-1)^m b^{(m-1)m/2}$.
%
%We consider the (infinite) set all possible generating matrices $C_1,C_2,\dots$. We also define the class 
%
%Indeed, we inherit all the group properties from the Permuatations group... because it is equivalent....

%\subsection{Owen Scrambling To Replicated Digital Sequences}
%
%It will be easy to write that we can apply an Owen scrambling to each dimension separately, and we only need to keep the same scrambling for $C_j$ and $C'_j$.


\subsection{Additive Approach}
%Notations:
%\begin{itemize}
%\item[.] $m$ denote the number of digits
%\item[.] $C_{m,1},\dots,C_{m,d}$: $d$ generator matrices of size $m \times m$ over the finite field $\mathcal{Z}_2$
%\item[.] $\vi$: base-$2$ representation of the integer $i$: $\vi=(i_1,\dots,i_m)^T$
%\item[.] $L_{m,l}$: nonsingular $m \times m$ lower triangular matrix over the finite field $\mathcal{Z}_2$
%\item[.] $\ve_{m,l}$: $m \times 1$ vector with elements from $\mathcal{Z}_2$
%\end{itemize}
%
%Denote by $X$ and $X'$ the two replicated designs of order $1$.
%A row of $X$ or $X'$ is a point in $[0,1]^d$. The symbol $x_i^j$ (resp. ${x'}_i^j$) corresponds to  the element of row $i$ and column $j$ of $X$ (resp. $X'$). The additive approach is described by Algorithm \ref{additive}. All operations are carried on the finite field $\mathcal{Z}_2$.
%\begin{algorithm}[!ht]
%
%\begin{center}
%\begin{minipage}{10cm}
%\begin{enumerate}
%\item[Step 1.] Instantiation: $X \leftarrow \emptyset$, $X' \leftarrow \emptyset$, $l \leftarrow 1$, $\widehat{\underline{\tau}_u^2}^{(0)} \leftarrow 0$.
%\item[Step 4.] $while$ ($!$ stopping criterion):
%\begin{enumerate}
%\item[4.1] Construct $L_{m,l}$ and $\ve_{m,l}$.
%\item[4.2] Augment both $X$ and $X'$:\\
%for $j=1,\dots,d$:\\
%for $i=1+2^{m+l-1},\dots,2^{m+l}$:
%\begin{flalign*}
%x_i^j & = C_{m,j} . \vi + \ve_{m,l} && \\
%{x'}_i^j & = L_{m,l} . (C_{m,j} . \vi +  \ve_{m,l})&&
%\end{flalign*}
%\item[4.3] For $u=1,\dots,d$: evaluate $\widehat{\underline{\tau}_u^2}^{(l)}.$ 
%\item[4.4.]  $l \leftarrow l+1.$
%\end{enumerate}
%\item[Step 5.] Return: $\widehat{\underline{\tau}_u^2}, \ u \in \{1,\dots,d\}.$ 
%\end{enumerate}
%\end{minipage}
%\end{center}
%\label{additive}
%\caption{Additive approach}
%\end{algorithm}

\end{document}