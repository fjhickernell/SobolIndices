% !TEX TS-program = PDFLatexBibtex
%&LaTeX
\documentclass[]{elsarticle}
\setlength{\marginparwidth}{0.5in}
\usepackage{amsmath,amssymb,amsthm,natbib,mathtools,bbm,extraipa,mathabx,graphicx}
\usepackage[ruled,vlined]{algorithm2e}
%accents,

\newtheorem{lem}{Lemma}
\newtheorem{remark}{Remark}
\theoremstyle{definition}
\newtheorem{defin}{Definition}
\newtheorem{algo}{Algorithm}

\newcommand{\fudge}{\fC}
\newcommand{\dtf}{\textit{\doubletilde{f}}}
\newcommand{\cube}{[0,1)^d}
%\renewcommand{\bbK}{\natzero^d}
\newcommand{\rf}{\mathring{f}}
\newcommand{\rnu}{\mathring{\nu}}
\newcommand{\natm}{\naturals_{0,m}}
\newcommand{\wcS}{\widecheck{S}}
\newcommand{\tol}{\text{tol}}
\newcommand{\e}{\text{e}}
\newcommand{\bvec}[1]{\boldsymbol{#1}}
\newcommand{\vx}{\bvec{x}}
\newcommand{\vi}{\bvec{i}}
\newcommand{\ve}{\bvec{e}}
\newcommand{\vk}{\bvec{k}}
\newcommand{\vz}{\bvec{z}}
\newcommand{\dif}{\mathsf{d}}
\newcommand{\hf}{\hat{f}}
\newcommand{\hS}{\widehat{S}}
\newcommand{\tS}{\widetilde{S}}
\newcommand{\tf}{\tilde{f}}
\newcommand{\fC}{\mathfrak{C}}
\newcommand{\homega}{\widehat{\omega}}
\newcommand{\wcomega}{\mathring{\omega}}
\newcommand{\vzero}{\bvec{0}}
\newcommand{\integers}{\mathbb{Z}}
\newcommand{\naturals}{\mathbb{N}}
\newcommand{\ip}[3][{}]{\ensuremath{\left \langle #2, #3 \right \rangle_{#1}}}

\def\abs#1{\ensuremath{\left \lvert #1 \right \rvert}}


\begin{document}

\begin{frontmatter}

\title{}

\author{Cl\'ementine Prieur, Elise Arnaud, Herv\'{e} Monod, Laurent Gilquin, Fred J. Hickernell, Llu\'{i}s Antoni Jim\'{e}nez Rugama}
\address{U. Josef Fourier, Illinois Institute of Technology}
\begin{abstract}
\end{abstract}

\end{frontmatter}

\section{Problem Statement}
Denote by $\vx \in [0,1)^d$ a point with components $x_1,\dots,x_d$. Let $u$ be a subset of $\{1,\dots,d\}$, $-u$ its complement and $|u|$ its cardinality. $\vx_u$ represents a point in $[0,1)^u$ with components $x_j, j \in u$. Given two points $\vx$ and $\vx'$, we define the following hybrid point: 
\[(\vx_u:{\vx'}_{-u}) := ({x'}_1,\dots,{x'}_{u-1},x_u,{x'}_{u+1},\dots,{x'}_d).\]
Let $f$ be the symbol representing the numerical model considered. We assume $f \in \mathcal{L}^2[0,1]^d$. Denote by $\mu$ and $\sigma^2$ the mean and variance of $f$.
%These two quantities can be expressed as the following integrals:
%\[\mu= \frac{1}{2} \int_{[0,1)^{2d-1}}f(\vx)+f(\vx_u:{\vx'}_{-u}) d\vx d{\vx'}_{-u}, \]
%\[\sigma^2= \int_{[0,1)^{d}}f(\vx)^2 d\vx - \Big{(}\int_{[0,1)^{d}}f(\vx) d\vx\Big{)}^2 .\]
The uncertainty on $\vx$ is modeled by random variables such that $\vx \sim \mathcal{U}[0,1]^d$. Recall the Hoeffding decomposition of $f$:
\begin{equation}
f(\vx)=\mu+\sum \limits_{u \subseteq \{1,\dots,d\}} f_u(\vx),
\label{anova}
\end{equation}
where:
\[f_u(\vx)= \int_{[0,1)^{|u|}} f(\vx) d{\vx}_{-u} - \sum \limits_{v \subset u} f_v(\vx).\]
Due to orthogonality, by taking the variance of each side in Equation \ref{anova} we obtain the following decomposition of the variance:
\[ \sigma^2 = \sum \limits_{u \subseteq \{1,\dots,d\}} \sigma_v^2.\]
The problem of interest is the estimation of the quantities:
\[\underline{\tau}_u^2 = \sum \limits_{v \subseteq u} \sigma_v^2, \qquad \text{ for } u = 1,\dots,d.\]
These quantities correspond to the partial sensitivity Sobol' indices often found written in the literature in their normalized form: $\underline{\tau}_u^2/\sigma^2$. When, $|u|=1$, $\underline{\tau}_u^2$ corresponds to the first-order Sobol' index that evaluate the main effect of the variable $\vx_u$. In \cite{}, Owen proposed to use the following expression for $\underline{\tau}_u^2$:
\[\underline{\tau}_u^2  =\int_{[0,1)^{2d-1}}(f(\vx) - \mu) (f(\vx_u:{\vx'}_{-u})-\mu) d\vx d{\vx'}_{-u},\]
taking $\mu$ as introduced in Janon \textit{et al.}:
\[\mu= \frac{1}{2} \int_{[0,1)^{2d-1}}f(\vx)+f(\vx_u:{\vx'}_{-u}) d\vx d{\vx'}_{-u}. \]
The natural way to estimate these two quantities is for $(\vx_i,\vx'_i) \stackrel{iid}{\sim} [0,1)^{2d}$ via:
\begin{align*}
\widehat{\underline{\tau}_u^2} & = \frac{1}{n} \sum \limits_{i=1}^n (f(\vx_i) - \mu) (f(\vx_{i,u}:{\vx'}_{i,-u})-\mu),\\
\widehat{\mu} & = \frac{1}{2n} \sum \limits_{i=1}^n f(\vx_i) +f(\vx_{i,u}:{\vx'}_{i,-u}).
\end{align*}
\bigskip

We consider here a quasi-Monte Carlo sampling strategy to evaluate these two estimators. We focus on the estimation of all first-order Sobol' indices. The classical estimation procedure introduced by Sobol' requires to evaluate the quantity $f(\vx_u:{\vx'}_{-u})$ $n$ times for each $u \in \{1,\dots,d\}$ while the quantity $f(\vx)$ is only evaluated $n$ times once for all $u$. The total cost sums up to $n(d+1)$ evaluations of $f$. This cost can rapidly becomes prohibitive for expansive models involving an important number of parameters. An improvement to reduces this cost lies in the use of replicated designs. The notion of replicated designs was first introduced by McKay. It is defined as follows:
\begin{defin}[Replicated designs]
Let $X$ and $X'$ be two designs defined as follow:
\[\arraycolsep=1.3pt
X=\left(\begin{array}{ccccc}
x_{1,1} & ... & x_{1,j} & ... & x_{1,d} \\
\vdots &  & \vdots & & \vdots \\
x_{i,1} & ... & x_{i,j} & ... & x_{i,d} \\
\vdots &  & \vdots & & \vdots \\
x_{n,1} & ... & x_{n,j} & ... & x_{n,d} \\
\end{array} \right), \ X'=\left(\begin{array}{ccccc}
{x'}_{1,1} & ... & {x'}_{1,j} & ... & {x'}_{1,d} \\
\vdots &  & \vdots & & \vdots \\
{x'}_{i,1} & ... & {x'}_{i,j} & ... & {x'}_{i,d} \\
\vdots &  & \vdots & & \vdots \\
{x'}_{n,1} & ... & {x'}_{n,j} & ... & {x'}_{n,d} \\
\end{array} \right).
\]
Let the symbol $X_u$ (resp. $X'_u$), $u \subsetneq \{1,\dots,d-1\}$, denotes a subset of $u$ columns of $X$ (resp. $X'$). We say that $X$ and $X'$ are two replicated designs of order $r \in \{1,\dots,d-1\}$ if:

$\forall \ u \subsetneq \{1,\dots,d\}$ such that $|u|=r$, $X_u$ and $X'_u$ represent the same point sets.
\end{defin}
The replication procedure introduced in \cite{} allows to estimate all first-order Sobol' indices with two replicated designs of order $1$. In this procedure, the quantity $f(\vx_u:{\vx'}_{-u})$ is only evaluated $n$ times once for all $u$ resulting in a total of $2n$ evaluations of $f$. Denote by $X$,$X'$ the two replicated designs of order $1$ required by the replication procedure, $X'$ is constructed from $X$ as follows:
\[{x'}_{i,j}=x_{\pi_j(i),j}, \qquad \forall \ (i,j) \in \{1,\dots,n\} \times \{1,\dots,d\},\]
where $\pi_1,\dots,\pi_d$ are $d$ permutations on $\{1,\dots,n\}$.
\bigskip

Our objective is to combine the use of Sobol' sequences with the replication method to iteratively estimate all quantities $\underline{\tau}_u^2, u \in \{1,\dots,d\}$. This would reduce the method to constructing two replicated designs of order $1$ that are iteratively augmented with new point sets. Two approaches are considered: one multiplicative and one additive. The first one uses the innate nested structure of a Sobol' sequence. The size of each replicated design is doubled at each step. The additive approach consists first of constructing a Sobol' sequence then shifting and scrambling it to form the new point sets. These new point sets are then added to the initial Sobol' sequence to form the two replicated designs. In this case, the size of each replicated design at step $k$ is $ k \times 2^m$ where $2^m$ is the size of the initial Sobol' sequence.



\section{Sobol' sequence background}

We are interested in digital nets which are constructed as point sets in $\cube$.

For $d\geq 1$, $b\geq 2$, and $0\geq t\geq m$, the set of points $\mathcal{P}\in\cube$, with $|\mathcal{P}|=b^m$, is a $(t,m,d)-net$ in base $b$ if for every $A\in\mathcal{A}$ with volume $b^{t-m}$, $|A\cap\mathcal{P}|=b^t$ ($A$ contains exactly $b^t$ points of $\mathcal{P}$).

A Sobol' sequence in dimension $d$ is a $(t,d)-sequence$ in base 2.

Obviously, we do not need to evaluate $f(\vx)$ $d$ times. But we can do event better than that, and evaluate $f(\vx_u:{\vx'}_{-u})$ only one time. This is the replication method....

$X$,$X'$ are two sequences such that,
\[
X = \begin{pmatrix}
x_1^1,\dots,x_1^d \\
x_2^1,\dots,x_2^d \\
x_3^1,\dots,x_3^d \\
\end{pmatrix} \\
X' = \begin{pmatrix}
\pi_1(x_1^1),\dots,\pi_d(x_1^d) \\
\pi_1(x_2^1),\dots,\pi_d(x_2^d) \\
\pi_1(x_3^1),\dots,\pi_d(x_3^d) \\
\end{pmatrix}
\]

These sequences can be generated indeed with Owen scramblings, or which is equivalent, matrix generators $C_i$

\section{Matrix generators group $U_m$}

Be $U_m(\mathbb{F}_b)$ the multiplicative group of all invertible triangular matrices of size $m\times m$ over $\mathbb{F}_b$ ($GF(b)$). We have that $\abs{U_m(\mathbb{F}_b)}=(b-1)^m b^{(m-1)m/2}$.

We consider the (infinite) set all possible generating matrices $C_1,C_2,\dots$. We also define the class 

Indeed, we inherit all the group properties from the Permuatations group... because it is equivalent....

Not all choices are possible, for instance order 2 and higher Sobol' indices.


\section{Additive Approach}
Notations:
\begin{itemize}
\item[.] $m$ denote the number of digits
\item[.] $C_{m,1},\dots,C_{m,d}$: $d$ generator matrices of size $m \times m$ over the finite field $\mathcal{Z}_2$
\item[.] $\vi$: base-$2$ representation of the integer $i$: $\vi=(i_1,\dots,i_m)^T$
\item[.] $L_{m,l}$: nonsingular $m \times m$ lower triangular matrix over the finite field $\mathcal{Z}_2$
\item[.] $\ve_{m,l}$: $m \times 1$ vector with elements from $\mathcal{Z}_2$
\end{itemize}

Denote by $X$ and $X'$ the two replicated designs of order $1$.
A row of $X$ or $X'$ is a point in $[0,1]^d$. The symbol $x_i^j$ (resp. ${x'}_i^j$) corresponds to  the element of row $i$ and column $j$ of $X$ (resp. $X'$). The additive approach is described by Algorithm \ref{additive}. All operations are carried on the finite field $\mathcal{Z}_2$.
\begin{algorithm}[!ht]

\begin{center}
\begin{minipage}{10cm}
\begin{enumerate}
\item[Step 1.] Instantiation: $X \leftarrow \emptyset$, $X' \leftarrow \emptyset$, $l \leftarrow 1$, $\widehat{\underline{\tau}_u^2}^{(0)} \leftarrow 0$.
\item[Step 4.] $while$ ($!$ stopping criterion):
\begin{enumerate}
\item[4.1] Construct $L_{m,l}$ and $\ve_{m,l}$.
\item[4.2] Augment both $X$ and $X'$:\\
for $j=1,\dots,d$:\\
for $i=1+2^{m+l-1},\dots,2^{m+l}$:
\begin{flalign*}
x_i^j & = C_{m,j} . \vi + \ve_{m,l} && \\
{x'}_i^j & = L_{m,l} . (C_{m,j} . \vi +  \ve_{m,l})&&
\end{flalign*}
\item[4.3] For $u=1,\dots,d$: evaluate $\widehat{\underline{\tau}_u^2}^{(l)}.$ 
\item[4.4.]  $l \leftarrow l+1.$
\end{enumerate}
\item[Step 5.] Return: $\widehat{\underline{\tau}_u^2}, \ u \in \{1,\dots,d\}.$ 
\end{enumerate}
\end{minipage}
\end{center}
\label{additive}
\caption{Additive approach}
\end{algorithm}



\section{L Matrices Fred Method Too Restrictive}


\end{document}