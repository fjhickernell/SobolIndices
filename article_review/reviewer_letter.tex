\documentclass[10pt,a4paper]{article}
\usepackage[utf8]{inputenc}
\usepackage[english]{babel}
\usepackage{amssymb,
mathtools,bm,extraipa,mathabx,graphicx,algorithm}
\usepackage{color}

\newcommand{\vect}[1]{\boldsymbol{\mathbf{#1}}}
\newcommand{\vk}{\vect{k}}
\newcommand{\vx}{\vect{x}}
\newcommand{\vI}{\vect{I}}
\newcommand{\hS}{\widehat{S}}
\newcommand{\tS}{\widetilde{S}}
\newcommand{\wcS}{\widecheck{S}}
\def\dashfill{\cleaders\hbox to 2em{-}\hfill}

\makeatletter
\newcommand{\ov}[1]{
  \m@th\overline{\mbox{#1}\raisebox{2mm}{}}
}

\begin{document}

\textcolor{blue}{Dear associated editor and reviewers,\\
we would like to thank you for the positive and constructive feedback. Below we have
compiled a detailed list of answers (blue text) to the issues raised in the review reports. In particular, in light of reviewer's two remark, the Asian option problem was deleted and replaced with a new ``real function''.\\
Best regards,\\
Laurent Gilquin and Llu\'{i}s Antoni Jim\'{e}nez Rugama}

\section*{Report on ``Reliable error estimation for Sobol' indices''}

\textbf{\large{Associate Editor comments}}
\vspace*{0.5cm}

\begin{itemize}
\item[1.] Page 4, left, line 52: What are $\kappa$ and $\vk(\kappa)$ there? They seem undefined, and then many readers will not understand what follows.
\item[2.] Page 4, right, line 30: Remove the period before (13).
\item[3.] Page 4, right, lines 38--42: We find $\epsilon_{\hat I}$ in (14) and $\epsilon_{I}$ below (seems inconsistent).
\item[4.] Section 5.2: These empirical results are for what parameters $d$, $T$, $S_0$, $\sigma$, $\rho$, $K$? I expect the behavior to depend strongly on the values of these parameters. E.g., large $d$ vs small $d$, large $T$ vs small $T$, large $\sigma$ vs small $\sigma$, etc. Also, this example is largely academic and not based on real-life data; I would not call it a "real-life" example. 
\item[5.] The given examples are only in small dimensions. What about higher-dimensional "real-life" examples?-- Pierre L'Ecuyer
\end{itemize}




\textbf{\large{Reviewer $1$}}
\vspace*{0.5cm}

\textbf{Major comments}:

\begin{itemize}
\item[1.] Advantages of using this method: The proposed method seems to work well in that with enough function evaluations, in most case the Sobol' indices can be estimated with good accuracy. In Section 2.3, the authors discuss a few alternative approaches and their disadvantages. They argue the main advantages of their approach are that 1) it guarantees the desired error threshold is met and 2) is not costly to compute. A concern with 1) is that we see later that sometimes the error threshold looks like it is not met (see item (3) below on the topic of the set C). A concern with 2) is that later in the paper, the discussion of the cost only focuses on the evaluation of the function f and seems to ignore the cost of constructing the error threshold (see item (2) below about "complete assessment of the cost"). In light of these concerns, simpler methods based on random sampling and straightforward confidence intervals become more attractive, as they have no extra cost and potentially a better assessment of how likely it is that the error is met (i.e., they provide a probabilistic statement about the error vs hoping parameters have been chosen so that the threshold is valid). I think this point should be discussed more thoroughly -- perhaps in the conclusion, or in Section 5 -- by the authors in order to convince the readers that the proposed method is indeed better.

\item[2.] Complete assessment of the cost: Since only the different variants of this method are compared in the numerical results, it is fair to focus on the number of function evaluations to compare their respective costs. However, and as discussed in paper [4] from the list of references (in particular toward the end of Theorem 1 in [4]), the overall cost of this method should also include the Walsh function evaluations required to determine the quantity $\widetilde{S}_{l,m}$ used in the error calculation via (14). The authors should mention this and discuss this impact of this cost on comparisons that are made with other methods.

\item[3.]The set C: In the section describing numerical results, choices are made for the parameters $\ell^*$ and $r$ which in turn define the set (cone) of functions C for which the error threshold formula (14) used to derive the algorithm holds. There should be some information in the paper on whether or not the functions considered in this section can be shown to be in this cone. This is especially important in light of the results obtained on page 10 (column 1, lines 36-38), where it looks like the error threshold was not met by the algorithm. Is it because the function is not in C?
\end{itemize}


\textbf{Minor comments}:
\begin{itemize}
\item[a.] page 1, column 1, lines 23-24: "through" instead of "trough"
\item[b.] page 2, column 2, equation (7): why use a biased estimator for $\sigma^2$ ?
\item[c.] page 4, column 2, equation (14): it is not clear how $\ell$ is chosen on the RHS, and whether or not the inequality holds for all $\ell \leq m$ or for some specific value of $\ell$ ? Please clarify.
\item[d.] page 4, column 2, line 42: is it not instead "...the error bound $\epsilon_{\hat{I}}$" (hat on subscript I?)?
\item[e.] page 5, column 2 line 55: not clear what you mean by "independent" in "two independent Sobol' sequences", especially if the designs are not randomized. From [4] it looks like you are using a 2d-dimensional Sobol' sequence and assign the first d coordinates to the first design and the next d ones to the second design. Please explain this more and state what properties you need for these sequences.
\item[f.] page 6, column 2, line 32: is the "iid" property valid across the indices i, i.e., the n vectors have to be independent? If so, this would invalidate the use of a Sobol' sequence to create these vectors, so please clarify this point.
\item[g.] page 9, column 2 , "real case model". There are some imprecisions in this part: lines 15-16, it should be the "discounted payoff" and not just "payoff", rho is the risk-free interest rate (not just interest rate). More importantly, the authors directly give the discounted payoff in terms of the Brownian motion without stating what model is used for the stock price and without stating that for the purpose of pricing one should work under the risk-neutral measure. Please clarify this so that the financial motivation for this problem is properly done.
\end{itemize}

\textbf{\large{Reviewer $2$}}
\vspace*{0.5cm}

\textbf{Major comments}:
\begin{itemize}
\item[1.] Definition 1 is not enough to get good points. Additional criteria should be attached to it, either by making the definition stricter or requiring some
other conditions to hold.

To illustrate, suppose that $u_0,\dots,u_{n-1} \in [0,1]$ are chosen. Then let $\vx_i = (u_i, u_i, \dots,u_i) \in \mathbb{R}^d$ and choose $\vx'_i= \vx_{\pi(i)}$ where $\pi$ is a permutation
of $0$ through $n-1$. Now we have replicated designs of order $a$ for any $a = 1,\dots,d-1$. In fact simply taking $n = 1$ and $\vx'=\vx \in [0,1]^d$ will do.

Also, how is $\pi_u$ defined when there is more than one permutation that rearranges the rows $\vx'_i$ into $\vx_i$? Maybe it should say ``to be any permutation
that rearranges the rows ... ''.

{\color{blue} TO REVIEW, THE COORDINATE-WISE DIFFERENT CLASS INCLUDES THE EXAMPLE HE PROPOSES --- We thank the reviewer for this remark. The definition was intended for point sets where all points are coordinate-wise different. More precisely, consider a point set $\mathcal{P}=\{\vx_i\}_{i=0}^{n-1}$ and two points $\vx_{i_1}$, $\vx_{i_2}$, $i_1 \neq i_2$. Then $\vx_{i_1}$ and $\vx_{i_2}$ are coordinate-wise different if: 
$$ x_{i_1,j} \neq x_{i_2,j}, \qquad \forall j=1,\dots,d .$$
Definition 1 has been reworked to include this property. This also tackles the issue of having more than one permutation that would rearrange the rows $\vx'_i$ into $\vx_i$. Now, the permutation is unique.

Furthermore, a remark was added to specify that in this paper $\mathcal{P}$ and $\mathcal{P}'$ are constructed to avoid the degenerate case: $\forall i, \ \vx'_i=\vx_i$.}



\item[2.] Section $3.1$ needs more detail for readers not already familiar with the approach. For instance on page $4$ line $16$ a little should be said about what $k_{j\ell}$ is. It should also briefly say in words what $S_m$, $\hS_{\ell,m}$, $\wcS_m$ and $\tS_{\ell,m}$ all mean.

\item[3.] Page $5$ line $56$ presents some bounds that the Sobol' indices must satisfy. Sometimes they won't satisfy those bounds. When do they fail, how that
be detectable, and what can we do in the face of bounds that might or might not hold? Presumably these issues have been discussed in simpler contexts. The paper should address this issue.

\item[4.]If we replace $f$ by $f-c$ for some constant shift $c$, then the Sobol' indices do not change. But the bounds for $\underline{S}_u$ and $\overline{S}_u$ do change because $\vI_3$ is affected. What could/should we do about this?

{\color {blue} We acknowledge that Sobol' indices are theoretically invariant by adding a constant shift $c$ to the studied function. As underlined by the reviewer, the addition of a constant shift $c$ modifies the error bound for $\vI_3$ but also the errors bounds for the others integrands ($\vI_1$, $\vI_2$ and $\vI_4$). More generally, our algorithm depends on the analytical expression of the Sobol' index estimator considered.

However, that is not a concerns since error bounds for any estimator can be calculated. That is, our algorithm can be adapted to any Sobol' index estimator. In particular, this is done when ``Correlation 2'' is used in Variant A.b instead of (5).}


\item[5.] Page $7$ lines $22:27$. The discussion here is saying that the higher dimensional integrand is likely to be the one that needs more evaluations. That is not in line with the last couple of decades of QMC theory. The nominal dimension might not be very telling. Maybe only a few of the inputs are important in the higher dimensional integrand, or maybe the higher order variable interactions are less important for that integrand.

So, can you give a better reason for why that integrand might need larger $n$? Alternatively, in the numerical examples, which integrands required the most evaluations?
(This point touches on the circular issue, perhaps best avoided, of finding which variables are important when you're estimating a Sobol' index.)

\item[6.] In Section $5.1.1$, the Sobol' g-function looks like a potentially misleading choice. It is completely smooth except when one of the inputs is $1/2$.
That makes it a particularly favorable problem for Sobol' points or any points in base $2$. The problem might just be too QMC-friendly. It would be better to use $g_j(x_j) = (|3x_j-2| + a_j)/(1 + a_j)$ which has issues when
$x_j = 2/3$ which is not dyadic. The QMC friendly version might still work exactly the same as the unfriendly version, but changing $g$ just removes that issue from consideration.

{\color{blue} The Sobol' g-function has been modified according to the reviewer remark. Corresponding tables and figures have changed.} 

\item[7.] The Asian option is very over used in QMC problems. We already know that BB and PC make the first component most important. It might be interesting to compare these two, although the answer might depend on
the strike price and volatility. 

For the same amount of work, the authors could use a real function where the answer would be interesting. Derek Bingham maintains a site of test functions for computer experiments. Some of those would be more suitable.

{\color{blue} The Asian option problem has been replaced by the wing weight test function taken from the suggested database. Section $5.2$ has been reworked to introduce this new test function. Our results are compared to formers obtained by screening approaches.} 

\item[8.] Sobol' indices are often motivated by the need to identify important variables when $f$ is expensive to evaluate; perhaps half a day for $n = 1$. Estimates that require $n$ to be several hundred thousand do not fitthat motivating context. If the authors can describe a context where it would be worth having $n = 400000$ to get a Sobol' estimate, that would be great. Nobody else has properly addressed this issue, so this point is a
suggestion, not a requirement.

{\color{blue} We understand and agree with the issue raised by the reviewer. A possible solution is to choose a higher target precision $\epsilon$ when dealing with time-expensive function. In the proposed examples, $\epsilon$ was intentionally fixed small to put forward the performance of our algorithm. To cover both situations (small and high value of $\epsilon$), the bratley example has been reworked.

On another hand, if one has to deal with a function where one day is required to obtain one evaluation, perhaps the use of Sobol' indices should be dropped for a screening method (less accurate but less greedy). Or if available, one could make use of a cluster to perform multiple model evaluations in parallel.}
\end{itemize}

\textbf{Minor comments}:
\begin{itemize}
\item[a.] page $2$ line $48$: it should be clear that $v \subset $u excludes $v = u$.

\item[b.] page $2$ line $59$: it is not clear why $u = \mathcal{D}$ is ruled out. In that case both $\underline{\tau}_u^2$ and $\ov{$\tau$}_u^2$ equal $\sigma^2$.

\item[c.] equation (7) is a numerically poor way to estimate $\sigma^2$ (roundoff errors).

\item[d.] page $3$ line $37$: accordingly $\rightarrow$ according

\item[e.] page $4$: there is a symbol class of $S$ for Sobol' indices with $S_m(f)$ and related quantities. The paper could note the distinction or possibly use a
new letter.

\item[f.] page $8$ line $16$: twice as small  $\rightarrow$ half as large
\end{itemize}

{\color{blue} All minors points have been dealt with. For point c., equation (7) was replaced with the unbiased estimator of $\sigma^2$. For point e., the distinction has been noted in the paper given that notations cannot be changed (they need to match those of the sourced paper).
\end{document}